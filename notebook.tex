
% Default to the notebook output style

    


% Inherit from the specified cell style.




    
\documentclass[11pt]{article}

    
    
    \usepackage[T1]{fontenc}
    % Nicer default font (+ math font) than Computer Modern for most use cases
    \usepackage{mathpazo}

    % Basic figure setup, for now with no caption control since it's done
    % automatically by Pandoc (which extracts ![](path) syntax from Markdown).
    \usepackage{graphicx}
    % We will generate all images so they have a width \maxwidth. This means
    % that they will get their normal width if they fit onto the page, but
    % are scaled down if they would overflow the margins.
    \makeatletter
    \def\maxwidth{\ifdim\Gin@nat@width>\linewidth\linewidth
    \else\Gin@nat@width\fi}
    \makeatother
    \let\Oldincludegraphics\includegraphics
    % Set max figure width to be 80% of text width, for now hardcoded.
    \renewcommand{\includegraphics}[1]{\Oldincludegraphics[width=.8\maxwidth]{#1}}
    % Ensure that by default, figures have no caption (until we provide a
    % proper Figure object with a Caption API and a way to capture that
    % in the conversion process - todo).
    \usepackage{caption}
    \DeclareCaptionLabelFormat{nolabel}{}
    \captionsetup{labelformat=nolabel}

    \usepackage{adjustbox} % Used to constrain images to a maximum size 
    \usepackage{xcolor} % Allow colors to be defined
    \usepackage{enumerate} % Needed for markdown enumerations to work
    \usepackage{geometry} % Used to adjust the document margins
    \usepackage{amsmath} % Equations
    \usepackage{amssymb} % Equations
    \usepackage{textcomp} % defines textquotesingle
    % Hack from http://tex.stackexchange.com/a/47451/13684:
    \AtBeginDocument{%
        \def\PYZsq{\textquotesingle}% Upright quotes in Pygmentized code
    }
    \usepackage{upquote} % Upright quotes for verbatim code
    \usepackage{eurosym} % defines \euro
    \usepackage[mathletters]{ucs} % Extended unicode (utf-8) support
    \usepackage[utf8x]{inputenc} % Allow utf-8 characters in the tex document
    \usepackage{fancyvrb} % verbatim replacement that allows latex
    \usepackage{grffile} % extends the file name processing of package graphics 
                         % to support a larger range 
    % The hyperref package gives us a pdf with properly built
    % internal navigation ('pdf bookmarks' for the table of contents,
    % internal cross-reference links, web links for URLs, etc.)
    \usepackage{hyperref}
    \usepackage{longtable} % longtable support required by pandoc >1.10
    \usepackage{booktabs}  % table support for pandoc > 1.12.2
    \usepackage[inline]{enumitem} % IRkernel/repr support (it uses the enumerate* environment)
    \usepackage[normalem]{ulem} % ulem is needed to support strikethroughs (\sout)
                                % normalem makes italics be italics, not underlines
    

    
    
    % Colors for the hyperref package
    \definecolor{urlcolor}{rgb}{0,.145,.698}
    \definecolor{linkcolor}{rgb}{.71,0.21,0.01}
    \definecolor{citecolor}{rgb}{.12,.54,.11}

    % ANSI colors
    \definecolor{ansi-black}{HTML}{3E424D}
    \definecolor{ansi-black-intense}{HTML}{282C36}
    \definecolor{ansi-red}{HTML}{E75C58}
    \definecolor{ansi-red-intense}{HTML}{B22B31}
    \definecolor{ansi-green}{HTML}{00A250}
    \definecolor{ansi-green-intense}{HTML}{007427}
    \definecolor{ansi-yellow}{HTML}{DDB62B}
    \definecolor{ansi-yellow-intense}{HTML}{B27D12}
    \definecolor{ansi-blue}{HTML}{208FFB}
    \definecolor{ansi-blue-intense}{HTML}{0065CA}
    \definecolor{ansi-magenta}{HTML}{D160C4}
    \definecolor{ansi-magenta-intense}{HTML}{A03196}
    \definecolor{ansi-cyan}{HTML}{60C6C8}
    \definecolor{ansi-cyan-intense}{HTML}{258F8F}
    \definecolor{ansi-white}{HTML}{C5C1B4}
    \definecolor{ansi-white-intense}{HTML}{A1A6B2}

    % commands and environments needed by pandoc snippets
    % extracted from the output of `pandoc -s`
    \providecommand{\tightlist}{%
      \setlength{\itemsep}{0pt}\setlength{\parskip}{0pt}}
    \DefineVerbatimEnvironment{Highlighting}{Verbatim}{commandchars=\\\{\}}
    % Add ',fontsize=\small' for more characters per line
    \newenvironment{Shaded}{}{}
    \newcommand{\KeywordTok}[1]{\textcolor[rgb]{0.00,0.44,0.13}{\textbf{{#1}}}}
    \newcommand{\DataTypeTok}[1]{\textcolor[rgb]{0.56,0.13,0.00}{{#1}}}
    \newcommand{\DecValTok}[1]{\textcolor[rgb]{0.25,0.63,0.44}{{#1}}}
    \newcommand{\BaseNTok}[1]{\textcolor[rgb]{0.25,0.63,0.44}{{#1}}}
    \newcommand{\FloatTok}[1]{\textcolor[rgb]{0.25,0.63,0.44}{{#1}}}
    \newcommand{\CharTok}[1]{\textcolor[rgb]{0.25,0.44,0.63}{{#1}}}
    \newcommand{\StringTok}[1]{\textcolor[rgb]{0.25,0.44,0.63}{{#1}}}
    \newcommand{\CommentTok}[1]{\textcolor[rgb]{0.38,0.63,0.69}{\textit{{#1}}}}
    \newcommand{\OtherTok}[1]{\textcolor[rgb]{0.00,0.44,0.13}{{#1}}}
    \newcommand{\AlertTok}[1]{\textcolor[rgb]{1.00,0.00,0.00}{\textbf{{#1}}}}
    \newcommand{\FunctionTok}[1]{\textcolor[rgb]{0.02,0.16,0.49}{{#1}}}
    \newcommand{\RegionMarkerTok}[1]{{#1}}
    \newcommand{\ErrorTok}[1]{\textcolor[rgb]{1.00,0.00,0.00}{\textbf{{#1}}}}
    \newcommand{\NormalTok}[1]{{#1}}
    
    % Additional commands for more recent versions of Pandoc
    \newcommand{\ConstantTok}[1]{\textcolor[rgb]{0.53,0.00,0.00}{{#1}}}
    \newcommand{\SpecialCharTok}[1]{\textcolor[rgb]{0.25,0.44,0.63}{{#1}}}
    \newcommand{\VerbatimStringTok}[1]{\textcolor[rgb]{0.25,0.44,0.63}{{#1}}}
    \newcommand{\SpecialStringTok}[1]{\textcolor[rgb]{0.73,0.40,0.53}{{#1}}}
    \newcommand{\ImportTok}[1]{{#1}}
    \newcommand{\DocumentationTok}[1]{\textcolor[rgb]{0.73,0.13,0.13}{\textit{{#1}}}}
    \newcommand{\AnnotationTok}[1]{\textcolor[rgb]{0.38,0.63,0.69}{\textbf{\textit{{#1}}}}}
    \newcommand{\CommentVarTok}[1]{\textcolor[rgb]{0.38,0.63,0.69}{\textbf{\textit{{#1}}}}}
    \newcommand{\VariableTok}[1]{\textcolor[rgb]{0.10,0.09,0.49}{{#1}}}
    \newcommand{\ControlFlowTok}[1]{\textcolor[rgb]{0.00,0.44,0.13}{\textbf{{#1}}}}
    \newcommand{\OperatorTok}[1]{\textcolor[rgb]{0.40,0.40,0.40}{{#1}}}
    \newcommand{\BuiltInTok}[1]{{#1}}
    \newcommand{\ExtensionTok}[1]{{#1}}
    \newcommand{\PreprocessorTok}[1]{\textcolor[rgb]{0.74,0.48,0.00}{{#1}}}
    \newcommand{\AttributeTok}[1]{\textcolor[rgb]{0.49,0.56,0.16}{{#1}}}
    \newcommand{\InformationTok}[1]{\textcolor[rgb]{0.38,0.63,0.69}{\textbf{\textit{{#1}}}}}
    \newcommand{\WarningTok}[1]{\textcolor[rgb]{0.38,0.63,0.69}{\textbf{\textit{{#1}}}}}
    
    
    % Define a nice break command that doesn't care if a line doesn't already
    % exist.
    \def\br{\hspace*{\fill} \\* }
    % Math Jax compatability definitions
    \def\gt{>}
    \def\lt{<}
    % Document parameters
    \title{FinalProject\_group78}
    
    
    

    % Pygments definitions
    
\makeatletter
\def\PY@reset{\let\PY@it=\relax \let\PY@bf=\relax%
    \let\PY@ul=\relax \let\PY@tc=\relax%
    \let\PY@bc=\relax \let\PY@ff=\relax}
\def\PY@tok#1{\csname PY@tok@#1\endcsname}
\def\PY@toks#1+{\ifx\relax#1\empty\else%
    \PY@tok{#1}\expandafter\PY@toks\fi}
\def\PY@do#1{\PY@bc{\PY@tc{\PY@ul{%
    \PY@it{\PY@bf{\PY@ff{#1}}}}}}}
\def\PY#1#2{\PY@reset\PY@toks#1+\relax+\PY@do{#2}}

\expandafter\def\csname PY@tok@w\endcsname{\def\PY@tc##1{\textcolor[rgb]{0.73,0.73,0.73}{##1}}}
\expandafter\def\csname PY@tok@c\endcsname{\let\PY@it=\textit\def\PY@tc##1{\textcolor[rgb]{0.25,0.50,0.50}{##1}}}
\expandafter\def\csname PY@tok@cp\endcsname{\def\PY@tc##1{\textcolor[rgb]{0.74,0.48,0.00}{##1}}}
\expandafter\def\csname PY@tok@k\endcsname{\let\PY@bf=\textbf\def\PY@tc##1{\textcolor[rgb]{0.00,0.50,0.00}{##1}}}
\expandafter\def\csname PY@tok@kp\endcsname{\def\PY@tc##1{\textcolor[rgb]{0.00,0.50,0.00}{##1}}}
\expandafter\def\csname PY@tok@kt\endcsname{\def\PY@tc##1{\textcolor[rgb]{0.69,0.00,0.25}{##1}}}
\expandafter\def\csname PY@tok@o\endcsname{\def\PY@tc##1{\textcolor[rgb]{0.40,0.40,0.40}{##1}}}
\expandafter\def\csname PY@tok@ow\endcsname{\let\PY@bf=\textbf\def\PY@tc##1{\textcolor[rgb]{0.67,0.13,1.00}{##1}}}
\expandafter\def\csname PY@tok@nb\endcsname{\def\PY@tc##1{\textcolor[rgb]{0.00,0.50,0.00}{##1}}}
\expandafter\def\csname PY@tok@nf\endcsname{\def\PY@tc##1{\textcolor[rgb]{0.00,0.00,1.00}{##1}}}
\expandafter\def\csname PY@tok@nc\endcsname{\let\PY@bf=\textbf\def\PY@tc##1{\textcolor[rgb]{0.00,0.00,1.00}{##1}}}
\expandafter\def\csname PY@tok@nn\endcsname{\let\PY@bf=\textbf\def\PY@tc##1{\textcolor[rgb]{0.00,0.00,1.00}{##1}}}
\expandafter\def\csname PY@tok@ne\endcsname{\let\PY@bf=\textbf\def\PY@tc##1{\textcolor[rgb]{0.82,0.25,0.23}{##1}}}
\expandafter\def\csname PY@tok@nv\endcsname{\def\PY@tc##1{\textcolor[rgb]{0.10,0.09,0.49}{##1}}}
\expandafter\def\csname PY@tok@no\endcsname{\def\PY@tc##1{\textcolor[rgb]{0.53,0.00,0.00}{##1}}}
\expandafter\def\csname PY@tok@nl\endcsname{\def\PY@tc##1{\textcolor[rgb]{0.63,0.63,0.00}{##1}}}
\expandafter\def\csname PY@tok@ni\endcsname{\let\PY@bf=\textbf\def\PY@tc##1{\textcolor[rgb]{0.60,0.60,0.60}{##1}}}
\expandafter\def\csname PY@tok@na\endcsname{\def\PY@tc##1{\textcolor[rgb]{0.49,0.56,0.16}{##1}}}
\expandafter\def\csname PY@tok@nt\endcsname{\let\PY@bf=\textbf\def\PY@tc##1{\textcolor[rgb]{0.00,0.50,0.00}{##1}}}
\expandafter\def\csname PY@tok@nd\endcsname{\def\PY@tc##1{\textcolor[rgb]{0.67,0.13,1.00}{##1}}}
\expandafter\def\csname PY@tok@s\endcsname{\def\PY@tc##1{\textcolor[rgb]{0.73,0.13,0.13}{##1}}}
\expandafter\def\csname PY@tok@sd\endcsname{\let\PY@it=\textit\def\PY@tc##1{\textcolor[rgb]{0.73,0.13,0.13}{##1}}}
\expandafter\def\csname PY@tok@si\endcsname{\let\PY@bf=\textbf\def\PY@tc##1{\textcolor[rgb]{0.73,0.40,0.53}{##1}}}
\expandafter\def\csname PY@tok@se\endcsname{\let\PY@bf=\textbf\def\PY@tc##1{\textcolor[rgb]{0.73,0.40,0.13}{##1}}}
\expandafter\def\csname PY@tok@sr\endcsname{\def\PY@tc##1{\textcolor[rgb]{0.73,0.40,0.53}{##1}}}
\expandafter\def\csname PY@tok@ss\endcsname{\def\PY@tc##1{\textcolor[rgb]{0.10,0.09,0.49}{##1}}}
\expandafter\def\csname PY@tok@sx\endcsname{\def\PY@tc##1{\textcolor[rgb]{0.00,0.50,0.00}{##1}}}
\expandafter\def\csname PY@tok@m\endcsname{\def\PY@tc##1{\textcolor[rgb]{0.40,0.40,0.40}{##1}}}
\expandafter\def\csname PY@tok@gh\endcsname{\let\PY@bf=\textbf\def\PY@tc##1{\textcolor[rgb]{0.00,0.00,0.50}{##1}}}
\expandafter\def\csname PY@tok@gu\endcsname{\let\PY@bf=\textbf\def\PY@tc##1{\textcolor[rgb]{0.50,0.00,0.50}{##1}}}
\expandafter\def\csname PY@tok@gd\endcsname{\def\PY@tc##1{\textcolor[rgb]{0.63,0.00,0.00}{##1}}}
\expandafter\def\csname PY@tok@gi\endcsname{\def\PY@tc##1{\textcolor[rgb]{0.00,0.63,0.00}{##1}}}
\expandafter\def\csname PY@tok@gr\endcsname{\def\PY@tc##1{\textcolor[rgb]{1.00,0.00,0.00}{##1}}}
\expandafter\def\csname PY@tok@ge\endcsname{\let\PY@it=\textit}
\expandafter\def\csname PY@tok@gs\endcsname{\let\PY@bf=\textbf}
\expandafter\def\csname PY@tok@gp\endcsname{\let\PY@bf=\textbf\def\PY@tc##1{\textcolor[rgb]{0.00,0.00,0.50}{##1}}}
\expandafter\def\csname PY@tok@go\endcsname{\def\PY@tc##1{\textcolor[rgb]{0.53,0.53,0.53}{##1}}}
\expandafter\def\csname PY@tok@gt\endcsname{\def\PY@tc##1{\textcolor[rgb]{0.00,0.27,0.87}{##1}}}
\expandafter\def\csname PY@tok@err\endcsname{\def\PY@bc##1{\setlength{\fboxsep}{0pt}\fcolorbox[rgb]{1.00,0.00,0.00}{1,1,1}{\strut ##1}}}
\expandafter\def\csname PY@tok@kc\endcsname{\let\PY@bf=\textbf\def\PY@tc##1{\textcolor[rgb]{0.00,0.50,0.00}{##1}}}
\expandafter\def\csname PY@tok@kd\endcsname{\let\PY@bf=\textbf\def\PY@tc##1{\textcolor[rgb]{0.00,0.50,0.00}{##1}}}
\expandafter\def\csname PY@tok@kn\endcsname{\let\PY@bf=\textbf\def\PY@tc##1{\textcolor[rgb]{0.00,0.50,0.00}{##1}}}
\expandafter\def\csname PY@tok@kr\endcsname{\let\PY@bf=\textbf\def\PY@tc##1{\textcolor[rgb]{0.00,0.50,0.00}{##1}}}
\expandafter\def\csname PY@tok@bp\endcsname{\def\PY@tc##1{\textcolor[rgb]{0.00,0.50,0.00}{##1}}}
\expandafter\def\csname PY@tok@fm\endcsname{\def\PY@tc##1{\textcolor[rgb]{0.00,0.00,1.00}{##1}}}
\expandafter\def\csname PY@tok@vc\endcsname{\def\PY@tc##1{\textcolor[rgb]{0.10,0.09,0.49}{##1}}}
\expandafter\def\csname PY@tok@vg\endcsname{\def\PY@tc##1{\textcolor[rgb]{0.10,0.09,0.49}{##1}}}
\expandafter\def\csname PY@tok@vi\endcsname{\def\PY@tc##1{\textcolor[rgb]{0.10,0.09,0.49}{##1}}}
\expandafter\def\csname PY@tok@vm\endcsname{\def\PY@tc##1{\textcolor[rgb]{0.10,0.09,0.49}{##1}}}
\expandafter\def\csname PY@tok@sa\endcsname{\def\PY@tc##1{\textcolor[rgb]{0.73,0.13,0.13}{##1}}}
\expandafter\def\csname PY@tok@sb\endcsname{\def\PY@tc##1{\textcolor[rgb]{0.73,0.13,0.13}{##1}}}
\expandafter\def\csname PY@tok@sc\endcsname{\def\PY@tc##1{\textcolor[rgb]{0.73,0.13,0.13}{##1}}}
\expandafter\def\csname PY@tok@dl\endcsname{\def\PY@tc##1{\textcolor[rgb]{0.73,0.13,0.13}{##1}}}
\expandafter\def\csname PY@tok@s2\endcsname{\def\PY@tc##1{\textcolor[rgb]{0.73,0.13,0.13}{##1}}}
\expandafter\def\csname PY@tok@sh\endcsname{\def\PY@tc##1{\textcolor[rgb]{0.73,0.13,0.13}{##1}}}
\expandafter\def\csname PY@tok@s1\endcsname{\def\PY@tc##1{\textcolor[rgb]{0.73,0.13,0.13}{##1}}}
\expandafter\def\csname PY@tok@mb\endcsname{\def\PY@tc##1{\textcolor[rgb]{0.40,0.40,0.40}{##1}}}
\expandafter\def\csname PY@tok@mf\endcsname{\def\PY@tc##1{\textcolor[rgb]{0.40,0.40,0.40}{##1}}}
\expandafter\def\csname PY@tok@mh\endcsname{\def\PY@tc##1{\textcolor[rgb]{0.40,0.40,0.40}{##1}}}
\expandafter\def\csname PY@tok@mi\endcsname{\def\PY@tc##1{\textcolor[rgb]{0.40,0.40,0.40}{##1}}}
\expandafter\def\csname PY@tok@il\endcsname{\def\PY@tc##1{\textcolor[rgb]{0.40,0.40,0.40}{##1}}}
\expandafter\def\csname PY@tok@mo\endcsname{\def\PY@tc##1{\textcolor[rgb]{0.40,0.40,0.40}{##1}}}
\expandafter\def\csname PY@tok@ch\endcsname{\let\PY@it=\textit\def\PY@tc##1{\textcolor[rgb]{0.25,0.50,0.50}{##1}}}
\expandafter\def\csname PY@tok@cm\endcsname{\let\PY@it=\textit\def\PY@tc##1{\textcolor[rgb]{0.25,0.50,0.50}{##1}}}
\expandafter\def\csname PY@tok@cpf\endcsname{\let\PY@it=\textit\def\PY@tc##1{\textcolor[rgb]{0.25,0.50,0.50}{##1}}}
\expandafter\def\csname PY@tok@c1\endcsname{\let\PY@it=\textit\def\PY@tc##1{\textcolor[rgb]{0.25,0.50,0.50}{##1}}}
\expandafter\def\csname PY@tok@cs\endcsname{\let\PY@it=\textit\def\PY@tc##1{\textcolor[rgb]{0.25,0.50,0.50}{##1}}}

\def\PYZbs{\char`\\}
\def\PYZus{\char`\_}
\def\PYZob{\char`\{}
\def\PYZcb{\char`\}}
\def\PYZca{\char`\^}
\def\PYZam{\char`\&}
\def\PYZlt{\char`\<}
\def\PYZgt{\char`\>}
\def\PYZsh{\char`\#}
\def\PYZpc{\char`\%}
\def\PYZdl{\char`\$}
\def\PYZhy{\char`\-}
\def\PYZsq{\char`\'}
\def\PYZdq{\char`\"}
\def\PYZti{\char`\~}
% for compatibility with earlier versions
\def\PYZat{@}
\def\PYZlb{[}
\def\PYZrb{]}
\makeatother


    % Exact colors from NB
    \definecolor{incolor}{rgb}{0.0, 0.0, 0.5}
    \definecolor{outcolor}{rgb}{0.545, 0.0, 0.0}



    
    % Prevent overflowing lines due to hard-to-break entities
    \sloppy 
    % Setup hyperref package
    \hypersetup{
      breaklinks=true,  % so long urls are correctly broken across lines
      colorlinks=true,
      urlcolor=urlcolor,
      linkcolor=linkcolor,
      citecolor=citecolor,
      }
    % Slightly bigger margins than the latex defaults
    
    \geometry{verbose,tmargin=1in,bmargin=1in,lmargin=1in,rmargin=1in}
    
    

    \begin{document}
    
    
    \maketitle
    
    

    
    \section{COGS 108 Final Project}\label{cogs-108-final-project}

    \section{Overview}\label{overview}

    For our project, we decided to see if we could predict the quality of
wine given its ingredients. Since wine is so easily and heavily
influenced by a number of factors that we can't control for (i.e
humidity, grape quality, etc), we wanted to focus on its internal
workings. We hypothesized that the volume of certain features, namely
alcohol percentage, volatile acidity and citric acid, were crucial in
determining a wine's quality rating. Using two datasets to test our
prediction, we found that these, as well as the other characteristics,
had no individual effect on overall rating, but rather, the rating was
derived from a combination of different features.

    \section{Names}\label{names}

\begin{itemize}
\tightlist
\item
  Alysha Ali
\item
  Tsai-Ying Ying Chuang
\item
  Zheng Zeng
\item
  Melody Yu
\item
  Ai-Ting Hsieh
\end{itemize}

    \section{Group Members IDs}\label{group-members-ids}

\begin{itemize}
\tightlist
\item
  A15345048
\item
  A15215376
\item
  A14679117
\item
  A14599481
\item
  A13595395
\end{itemize}

    \section{Research Question}\label{research-question}

    Can we predict the quality of a wine given a list of its physicochemical
features?

    \section{Background and Prior Work}\label{background-and-prior-work}

    Since its creation 6000 years ago, wine has been adapted widely --- from
an instrument for religious ceremonies to a bargaining tool between
countries (https://vinepair.com/wine-colonized-world-wine-history/\#5).
Despite the variety in usage, wine has constantly been regarded as a
social device and has only risen in popularity
(https://www.svb.com/globalassets/library/images/svb-2018-wine-report.pdf).
That's why, with the combination of the quickly-approaching legal
drinking age for several of our group members and its increasing
appearance in social
media(https://www.ncbi.nlm.nih.gov/pmc/articles/PMC4432862/), we thought
it was important to determine how the quality of a wine is determined.
Specifically, we aim to discover exactly what ingredients are crucial in
deriving a wine's quality rating.

We came to our hypothesis by first researching wine characteristics
(https://winefolly.com/review/wine-characteristics/). From there, we saw
that certain elements, like citric acid content and volatile acidity,
directly influenced a wine's sweetness. From there, we looked at ranges
and eliminated factors like water density, which only ranged from 0.99
to 1.00 and therefore could not have as large an impact on quality as
other factors, like total sulfur dioxide for example, which had a range
of over 200.

This information is important for all consumers, from pre-teens getting
their first sip of wine to experienced connoisseurs and businesses. With
this data analysis, instead of spending hours trying to choose a wine
everyone will love, party hosts and other entertainers will know what
the most favored wine features are and be able to buy a delicious spirit
in a snap. For instance, knowing what chemicals make a wine popular will
reduce how much time winemakers spend in perfecting their formula and
ultimately maximize efficiency and increase profit in the long run.
Thus, this information can affect more than just consumers.

Since these datasets were previously the subject of a competition hosted
by the University of California, Irvine, there have been several other
projects that have analyzed similar questions and/or have used the same
datasets. Most notably, is another study that also attempted to predict
the quality of wine
(https://www.kaggle.com/vishalyo990/prediction-of-quality-of-wine).
However, they differed in their methods used and the overall accuracy of
their machine learning model.

    \section{Hypothesis}\label{hypothesis}

    We hypothesize that we can predict the quality of the wine by its
individual features. Especially, we predict that the lower volatile
acidity, medium alcohol percentage, and higher citric acid will create
the best quality of the wine.

    \section{Dataset(s)}\label{datasets}

    To answer our question, we will be using two datasets: one on red wines
and one on white wines (source:
https://archive.ics.uci.edu/ml/datasets/wine+quality ). The red wine
dataset had 1601 observations, while the white wine dataset had 4900.
These datasets are identical in the features they analyze, namely a list
of physiochemical ingredients in wine:

\begin{itemize}
\item
  fixed acidity - most acids involved with wine or fixed or nonvolatile
  (do not evaporate readily)
\item
  volatile acidity - the amount of acetic acid in wine, which at too
  high of levels can lead to an unpleasant, vinegar taste
\item
  citric acid - found in small quantities, citric acid can add
  'freshness' and flavor to wines
\item
  residual sugar - the amount of sugar remaining after fermentation
  stops, it's rare to find wines with less than 1 gram/liter and wines
  with greater than 45 grams/liter are considered sweet
\item
  chlorides - the amount of salt in the wine
\item
  free sulfur dioxide - the free form of SO2 exists in equilibrium
  between molecular SO2 (as a dissolved gas) and bisulfite ion; it
  prevents microbial growth and the oxidation of wine
\item
  total sulfur dioxide - amount of free and bound forms of S02; in low
  concentrations, SO2 is mostly undetectable in wine, but at free SO2
  concentrations over 50 ppm, SO2 becomes evident in the nose and taste
  of wine
\item
  density - the density of water is close to that of water depending on
  the percent alcohol and sugar content
\item
  pH - describes how acidic or basic a wine is on a scale from 0 (very
  acidic) to 14 (very basic); most wines are between 3-4 on the pH scale
\item
  sulphates - a wine additive which can contribute to sulfur dioxide gas
  (S02) levels, wich acts as an antimicrobial and antioxidant
\item
  alcohol - the percent alcohol content of the wine
\item
  quality - output variable (based on sensory data, score between 0 and
  10)
\end{itemize}

We combined the two in several ways during our process. For instance, in
one part, we began by graphing the individual features together in order
to see a comparison between white and red wines. In another part, we
used random samples from both datasets in order to train and test our
algorithm.

    \section{Setup}\label{setup}

    \begin{Verbatim}[commandchars=\\\{\}]
{\color{incolor}In [{\color{incolor}4}]:} \PY{k+kn}{import} \PY{n+nn}{pandas} \PY{k}{as} \PY{n+nn}{pd}
        \PY{k+kn}{import} \PY{n+nn}{numpy} \PY{k}{as} \PY{n+nn}{np}
        \PY{k+kn}{import} \PY{n+nn}{pylab}
        \PY{k+kn}{import} \PY{n+nn}{requests}
        \PY{k+kn}{import} \PY{n+nn}{bs4}
        \PY{k+kn}{from} \PY{n+nn}{bs4} \PY{k}{import} \PY{n}{BeautifulSoup}
        \PY{k+kn}{import} \PY{n+nn}{matplotlib}\PY{n+nn}{.}\PY{n+nn}{pyplot} \PY{k}{as} \PY{n+nn}{plt}
        \PY{k+kn}{from} \PY{n+nn}{sklearn}\PY{n+nn}{.}\PY{n+nn}{neighbors} \PY{k}{import} \PY{n}{KDTree}
        \PY{k+kn}{from} \PY{n+nn}{sklearn}\PY{n+nn}{.}\PY{n+nn}{utils} \PY{k}{import} \PY{n}{shuffle}
        \PY{k+kn}{from} \PY{n+nn}{pandas}\PY{n+nn}{.}\PY{n+nn}{plotting} \PY{k}{import} \PY{n}{scatter\PYZus{}matrix}
\end{Verbatim}


    \begin{Verbatim}[commandchars=\\\{\}]
{\color{incolor}In [{\color{incolor}5}]:} \PY{c+c1}{\PYZsh{}Here we import the two datasets that we will be usiing in our final project}
        \PY{n}{df\PYZus{}red} \PY{o}{=} \PY{n}{pd}\PY{o}{.}\PY{n}{read\PYZus{}csv}\PY{p}{(}\PY{l+s+s1}{\PYZsq{}}\PY{l+s+s1}{wineQualityReds.csv}\PY{l+s+s1}{\PYZsq{}}\PY{p}{)}
        \PY{n}{df\PYZus{}white} \PY{o}{=} \PY{n}{pd}\PY{o}{.}\PY{n}{read\PYZus{}csv}\PY{p}{(}\PY{l+s+s1}{\PYZsq{}}\PY{l+s+s1}{wineQualityWhites.csv}\PY{l+s+s1}{\PYZsq{}}\PY{p}{)}
        
        \PY{c+c1}{\PYZsh{} print out the column names}
        \PY{n+nb}{print}\PY{p}{(}\PY{n}{df\PYZus{}red}\PY{o}{.}\PY{n}{columns}\PY{o}{.}\PY{n}{values}\PY{p}{)}
\end{Verbatim}


    \begin{Verbatim}[commandchars=\\\{\}]
['Unnamed: 0' 'fixed.acidity' 'volatile.acidity' 'citric.acid'
 'residual.sugar' 'chlorides' 'free.sulfur.dioxide' 'total.sulfur.dioxide'
 'density' 'pH' 'sulphates' 'alcohol' 'quality']

    \end{Verbatim}

    \section{Data Cleaning}\label{data-cleaning}

    Here, we will clean up our data before we start analyzing. We will
remove rows with missing data, and delete columns that are not useful.
More explanation will be included as we clean up the data.

    \begin{Verbatim}[commandchars=\\\{\}]
{\color{incolor}In [{\color{incolor}6}]:} \PY{n}{df\PYZus{}red}\PY{p}{[}\PY{p}{:}\PY{l+m+mi}{5}\PY{p}{]}
\end{Verbatim}


\begin{Verbatim}[commandchars=\\\{\}]
{\color{outcolor}Out[{\color{outcolor}6}]:}    Unnamed: 0  fixed.acidity  volatile.acidity  citric.acid  residual.sugar  \textbackslash{}
        0           1            7.4              0.70         0.00             1.9   
        1           2            7.8              0.88         0.00             2.6   
        2           3            7.8              0.76         0.04             2.3   
        3           4           11.2              0.28         0.56             1.9   
        4           5            7.4              0.70         0.00             1.9   
        
           chlorides  free.sulfur.dioxide  total.sulfur.dioxide  density    pH  \textbackslash{}
        0      0.076                 11.0                  34.0   0.9978  3.51   
        1      0.098                 25.0                  67.0   0.9968  3.20   
        2      0.092                 15.0                  54.0   0.9970  3.26   
        3      0.075                 17.0                  60.0   0.9980  3.16   
        4      0.076                 11.0                  34.0   0.9978  3.51   
        
           sulphates  alcohol  quality  
        0       0.56      9.4        5  
        1       0.68      9.8        5  
        2       0.65      9.8        5  
        3       0.58      9.8        6  
        4       0.56      9.4        5  
\end{Verbatim}
            
    \begin{Verbatim}[commandchars=\\\{\}]
{\color{incolor}In [{\color{incolor}7}]:} \PY{n}{df\PYZus{}red}\PY{o}{.}\PY{n}{shape}
\end{Verbatim}


\begin{Verbatim}[commandchars=\\\{\}]
{\color{outcolor}Out[{\color{outcolor}7}]:} (1599, 13)
\end{Verbatim}
            
    Drop any row with missing values

    \begin{Verbatim}[commandchars=\\\{\}]
{\color{incolor}In [{\color{incolor}8}]:} \PY{n}{df\PYZus{}red} \PY{o}{=} \PY{n}{df\PYZus{}red}\PY{o}{.}\PY{n}{dropna}\PY{p}{(}\PY{p}{)}
\end{Verbatim}


    Drop the column labeled "Unnamed: 0"

    \begin{Verbatim}[commandchars=\\\{\}]
{\color{incolor}In [{\color{incolor}9}]:} \PY{n}{df\PYZus{}red} \PY{o}{=} \PY{n}{df\PYZus{}red}\PY{o}{.}\PY{n}{drop}\PY{p}{(}\PY{p}{[}\PY{l+s+s1}{\PYZsq{}}\PY{l+s+s1}{Unnamed: 0}\PY{l+s+s1}{\PYZsq{}}\PY{p}{]}\PY{p}{,} \PY{n}{axis}\PY{o}{=}\PY{l+m+mi}{1}\PY{p}{)}
\end{Verbatim}


    \begin{Verbatim}[commandchars=\\\{\}]
{\color{incolor}In [{\color{incolor}10}]:} \PY{n}{df\PYZus{}red}\PY{o}{.}\PY{n}{shape}
\end{Verbatim}


\begin{Verbatim}[commandchars=\\\{\}]
{\color{outcolor}Out[{\color{outcolor}10}]:} (1599, 12)
\end{Verbatim}
            
    \begin{Verbatim}[commandchars=\\\{\}]
{\color{incolor}In [{\color{incolor}11}]:} \PY{n}{df\PYZus{}red}\PY{p}{[}\PY{p}{:}\PY{l+m+mi}{5}\PY{p}{]}
\end{Verbatim}


\begin{Verbatim}[commandchars=\\\{\}]
{\color{outcolor}Out[{\color{outcolor}11}]:}    fixed.acidity  volatile.acidity  citric.acid  residual.sugar  chlorides  \textbackslash{}
         0            7.4              0.70         0.00             1.9      0.076   
         1            7.8              0.88         0.00             2.6      0.098   
         2            7.8              0.76         0.04             2.3      0.092   
         3           11.2              0.28         0.56             1.9      0.075   
         4            7.4              0.70         0.00             1.9      0.076   
         
            free.sulfur.dioxide  total.sulfur.dioxide  density    pH  sulphates  \textbackslash{}
         0                 11.0                  34.0   0.9978  3.51       0.56   
         1                 25.0                  67.0   0.9968  3.20       0.68   
         2                 15.0                  54.0   0.9970  3.26       0.65   
         3                 17.0                  60.0   0.9980  3.16       0.58   
         4                 11.0                  34.0   0.9978  3.51       0.56   
         
            alcohol  quality  
         0      9.4        5  
         1      9.8        5  
         2      9.8        5  
         3      9.8        6  
         4      9.4        5  
\end{Verbatim}
            
    \begin{Verbatim}[commandchars=\\\{\}]
{\color{incolor}In [{\color{incolor}12}]:} \PY{n}{df\PYZus{}white}\PY{p}{[}\PY{p}{:}\PY{l+m+mi}{5}\PY{p}{]}
\end{Verbatim}


\begin{Verbatim}[commandchars=\\\{\}]
{\color{outcolor}Out[{\color{outcolor}12}]:}    Unnamed: 0  fixed.acidity  volatile.acidity  citric.acid  residual.sugar  \textbackslash{}
         0           1            7.0              0.27         0.36            20.7   
         1           2            6.3              0.30         0.34             1.6   
         2           3            8.1              0.28         0.40             6.9   
         3           4            7.2              0.23         0.32             8.5   
         4           5            7.2              0.23         0.32             8.5   
         
            chlorides  free.sulfur.dioxide  total.sulfur.dioxide  density    pH  \textbackslash{}
         0      0.045                 45.0                 170.0   1.0010  3.00   
         1      0.049                 14.0                 132.0   0.9940  3.30   
         2      0.050                 30.0                  97.0   0.9951  3.26   
         3      0.058                 47.0                 186.0   0.9956  3.19   
         4      0.058                 47.0                 186.0   0.9956  3.19   
         
            sulphates  alcohol  quality  
         0       0.45      8.8        6  
         1       0.49      9.5        6  
         2       0.44     10.1        6  
         3       0.40      9.9        6  
         4       0.40      9.9        6  
\end{Verbatim}
            
    \begin{Verbatim}[commandchars=\\\{\}]
{\color{incolor}In [{\color{incolor}13}]:} \PY{n}{df\PYZus{}white}\PY{o}{.}\PY{n}{shape}
\end{Verbatim}


\begin{Verbatim}[commandchars=\\\{\}]
{\color{outcolor}Out[{\color{outcolor}13}]:} (4898, 13)
\end{Verbatim}
            
    Drop any rows with missing values

    \begin{Verbatim}[commandchars=\\\{\}]
{\color{incolor}In [{\color{incolor}14}]:} \PY{n}{df\PYZus{}white} \PY{o}{=} \PY{n}{df\PYZus{}white}\PY{o}{.}\PY{n}{dropna}\PY{p}{(}\PY{p}{)}
\end{Verbatim}


    Drop the column labeled as "Unnamed: 0"

    \begin{Verbatim}[commandchars=\\\{\}]
{\color{incolor}In [{\color{incolor}15}]:} \PY{n}{df\PYZus{}white} \PY{o}{=} \PY{n}{df\PYZus{}white}\PY{o}{.}\PY{n}{drop}\PY{p}{(}\PY{p}{[}\PY{l+s+s1}{\PYZsq{}}\PY{l+s+s1}{Unnamed: 0}\PY{l+s+s1}{\PYZsq{}}\PY{p}{]}\PY{p}{,} \PY{n}{axis}\PY{o}{=}\PY{l+m+mi}{1}\PY{p}{)}
\end{Verbatim}


    \begin{Verbatim}[commandchars=\\\{\}]
{\color{incolor}In [{\color{incolor}16}]:} \PY{n}{df\PYZus{}white}\PY{o}{.}\PY{n}{shape}
\end{Verbatim}


\begin{Verbatim}[commandchars=\\\{\}]
{\color{outcolor}Out[{\color{outcolor}16}]:} (4898, 12)
\end{Verbatim}
            
    \begin{Verbatim}[commandchars=\\\{\}]
{\color{incolor}In [{\color{incolor}17}]:} \PY{n}{df\PYZus{}white}\PY{p}{[}\PY{p}{:}\PY{l+m+mi}{5}\PY{p}{]}
\end{Verbatim}


\begin{Verbatim}[commandchars=\\\{\}]
{\color{outcolor}Out[{\color{outcolor}17}]:}    fixed.acidity  volatile.acidity  citric.acid  residual.sugar  chlorides  \textbackslash{}
         0            7.0              0.27         0.36            20.7      0.045   
         1            6.3              0.30         0.34             1.6      0.049   
         2            8.1              0.28         0.40             6.9      0.050   
         3            7.2              0.23         0.32             8.5      0.058   
         4            7.2              0.23         0.32             8.5      0.058   
         
            free.sulfur.dioxide  total.sulfur.dioxide  density    pH  sulphates  \textbackslash{}
         0                 45.0                 170.0   1.0010  3.00       0.45   
         1                 14.0                 132.0   0.9940  3.30       0.49   
         2                 30.0                  97.0   0.9951  3.26       0.44   
         3                 47.0                 186.0   0.9956  3.19       0.40   
         4                 47.0                 186.0   0.9956  3.19       0.40   
         
            alcohol  quality  
         0      8.8        6  
         1      9.5        6  
         2     10.1        6  
         3      9.9        6  
         4      9.9        6  
\end{Verbatim}
            
    We want to predict wine quality based on their ingredients therefore the
information we need for each wine instance is the list of ingredient
values and the quality. In the above cells, we first dropped any rows
with missing values to ensure that we are using a dataset that contains
exactly what we need. Since this is already a very clean dataset, none
of the rows were dropped. The column labeled "Unnamed: 0" does not seem
to be useful at all since we also have the index of each row. Therefore,
we dropped the entire column. Now, the resulting dataframe only contains
information we need.

    Since the white wine dataset contains more instances, we will only take
1599 instances from it randomly to match the number of instances of red
wine.

    \begin{Verbatim}[commandchars=\\\{\}]
{\color{incolor}In [{\color{incolor}18}]:} \PY{n}{df\PYZus{}white} \PY{o}{=} \PY{n}{shuffle}\PY{p}{(}\PY{n}{df\PYZus{}white}\PY{p}{)}\PY{o}{.}\PY{n}{reset\PYZus{}index}\PY{p}{(}\PY{n}{drop} \PY{o}{=} \PY{k+kc}{True}\PY{p}{)}
         
         \PY{n}{df\PYZus{}white} \PY{o}{=} \PY{n}{df\PYZus{}white}\PY{p}{[}\PY{p}{:}\PY{l+m+mi}{1599}\PY{p}{]}
\end{Verbatim}


    Here we will combine two wine sets.

    \begin{Verbatim}[commandchars=\\\{\}]
{\color{incolor}In [{\color{incolor}19}]:} \PY{c+c1}{\PYZsh{} appending datasets of both white wine and red wine }
         \PY{n}{df\PYZus{}combined} \PY{o}{=} \PY{n}{df\PYZus{}red}\PY{o}{.}\PY{n}{append}\PY{p}{(}\PY{n}{df\PYZus{}white}\PY{p}{,} \PY{n}{ignore\PYZus{}index} \PY{o}{=}\PY{k+kc}{True}\PY{p}{)}
         \PY{n}{df\PYZus{}combined} \PY{o}{=} \PY{n}{shuffle}\PY{p}{(}\PY{n}{df\PYZus{}combined}\PY{p}{)}
         \PY{n}{df\PYZus{}combined} \PY{o}{=} \PY{n}{df\PYZus{}combined}\PY{o}{.}\PY{n}{reset\PYZus{}index}\PY{p}{(}\PY{n}{drop}\PY{o}{=}\PY{k+kc}{True}\PY{p}{)}
         
         \PY{n}{df\PYZus{}combined}\PY{o}{.}\PY{n}{shape}
\end{Verbatim}


\begin{Verbatim}[commandchars=\\\{\}]
{\color{outcolor}Out[{\color{outcolor}19}]:} (3198, 12)
\end{Verbatim}
            
    \begin{Verbatim}[commandchars=\\\{\}]
{\color{incolor}In [{\color{incolor}20}]:} \PY{n}{desc} \PY{o}{=} \PY{n}{df\PYZus{}combined}\PY{o}{.}\PY{n}{describe}\PY{p}{(}\PY{p}{)}
         \PY{n}{desc}
\end{Verbatim}


\begin{Verbatim}[commandchars=\\\{\}]
{\color{outcolor}Out[{\color{outcolor}20}]:}        fixed.acidity  volatile.acidity  citric.acid  residual.sugar  \textbackslash{}
         count    3198.000000       3198.000000  3198.000000     3198.000000   
         mean        7.593340          0.403094     0.302489        4.489290   
         std         1.549998          0.192557     0.164095        4.192491   
         min         4.200000          0.080000     0.000000        0.600000   
         25\%         6.600000          0.250000     0.210000        1.900000   
         50\%         7.200000          0.360000     0.300000        2.400000   
         75\%         8.200000          0.530000     0.400000        6.000000   
         max        15.900000          1.580000     1.000000       31.600000   
         
                  chlorides  free.sulfur.dioxide  total.sulfur.dioxide      density  \textbackslash{}
         count  3198.000000          3198.000000           3198.000000  3198.000000   
         mean      0.066540            25.200281             91.969669     0.995411   
         std       0.042143            16.418179             59.627286     0.002819   
         min       0.012000             1.000000              6.000000     0.987420   
         25\%       0.043000            12.000000             37.000000     0.993443   
         50\%       0.059000            22.000000             88.000000     0.995800   
         75\%       0.080000            35.000000            136.000000     0.997400   
         max       0.611000           112.000000            366.500000     1.010300   
         
                         pH    sulphates      alcohol      quality  
         count  3198.000000  3198.000000  3198.000000  3198.000000  
         mean      3.249912     0.572745    10.461024     5.761101  
         std       0.163454     0.167567     1.151039     0.851406  
         min       2.720000     0.230000     8.000000     3.000000  
         25\%       3.140000     0.460000     9.500000     5.000000  
         50\%       3.240000     0.550000    10.200000     6.000000  
         75\%       3.360000     0.650000    11.200000     6.000000  
         max       4.010000     2.000000    14.900000     8.000000  
\end{Verbatim}
            
    \begin{Verbatim}[commandchars=\\\{\}]
{\color{incolor}In [{\color{incolor}21}]:} \PY{n}{corrs} \PY{o}{=} \PY{n}{df\PYZus{}combined}\PY{o}{.}\PY{n}{corr}\PY{p}{(}\PY{p}{)}
         \PY{n}{corrs}
\end{Verbatim}


\begin{Verbatim}[commandchars=\\\{\}]
{\color{outcolor}Out[{\color{outcolor}21}]:}                       fixed.acidity  volatile.acidity  citric.acid  \textbackslash{}
         fixed.acidity              1.000000          0.171727     0.407795   
         volatile.acidity           0.171727          1.000000    -0.458312   
         citric.acid                0.407795         -0.458312     1.000000   
         residual.sugar            -0.164296         -0.281011     0.164973   
         chlorides                  0.296037          0.361918     0.059043   
         free.sulfur.dioxide       -0.334627         -0.397302     0.132845   
         total.sulfur.dioxide      -0.372436         -0.454174     0.197656   
         density                    0.556108          0.324357     0.111073   
         pH                        -0.292510          0.331062    -0.424707   
         sulphates                  0.345303          0.200511     0.113168   
         alcohol                   -0.090773         -0.095184     0.034802   
         quality                   -0.043420         -0.333890     0.158044   
         
                               residual.sugar  chlorides  free.sulfur.dioxide  \textbackslash{}
         fixed.acidity              -0.164296   0.296037            -0.334627   
         volatile.acidity           -0.281011   0.361918            -0.397302   
         citric.acid                 0.164973   0.059043             0.132845   
         residual.sugar              1.000000  -0.195536             0.489241   
         chlorides                  -0.195536   1.000000            -0.257042   
         free.sulfur.dioxide         0.489241  -0.257042             1.000000   
         total.sulfur.dioxide        0.555424  -0.326883             0.768073   
         density                     0.349580   0.376899            -0.092566   
         pH                         -0.293666   0.018465            -0.180862   
         sulphates                  -0.236341   0.462941            -0.243341   
         alcohol                    -0.280145  -0.227448            -0.152875   
         quality                     0.023986  -0.193329             0.090369   
         
                               total.sulfur.dioxide   density        pH  sulphates  \textbackslash{}
         fixed.acidity                    -0.372436  0.556108 -0.292510   0.345303   
         volatile.acidity                 -0.454174  0.324357  0.331062   0.200511   
         citric.acid                       0.197656  0.111073 -0.424707   0.113168   
         residual.sugar                    0.555424  0.349580 -0.293666  -0.236341   
         chlorides                        -0.326883  0.376899  0.018465   0.462941   
         free.sulfur.dioxide               0.768073 -0.092566 -0.180862  -0.243341   
         total.sulfur.dioxide              1.000000 -0.136670 -0.296915  -0.333072   
         density                          -0.136670  1.000000  0.022438   0.326821   
         pH                               -0.296915  0.022438  1.000000   0.139735   
         sulphates                        -0.333072  0.326821  0.139735   1.000000   
         alcohol                          -0.202789 -0.618483  0.144421   0.017505   
         quality                           0.007226 -0.282348 -0.014176   0.066996   
         
                                alcohol   quality  
         fixed.acidity        -0.090773 -0.043420  
         volatile.acidity     -0.095184 -0.333890  
         citric.acid           0.034802  0.158044  
         residual.sugar       -0.280145  0.023986  
         chlorides            -0.227448 -0.193329  
         free.sulfur.dioxide  -0.152875  0.090369  
         total.sulfur.dioxide -0.202789  0.007226  
         density              -0.618483 -0.282348  
         pH                    0.144421 -0.014176  
         sulphates             0.017505  0.066996  
         alcohol               1.000000  0.439750  
         quality               0.439750  1.000000  
\end{Verbatim}
            
    By reading the above correlation table, it seems like none of the
feature correlates with the wine's quality, we will plot out some
matrices to see if that really is the case.

    \begin{Verbatim}[commandchars=\\\{\}]
{\color{incolor}In [{\color{incolor}22}]:} \PY{n}{scatter\PYZus{}matrix}\PY{p}{(}\PY{n}{df\PYZus{}combined}\PY{p}{[}\PY{p}{[}\PY{l+s+s1}{\PYZsq{}}\PY{l+s+s1}{fixed.acidity}\PY{l+s+s1}{\PYZsq{}}\PY{p}{,}\PY{l+s+s1}{\PYZsq{}}\PY{l+s+s1}{quality}\PY{l+s+s1}{\PYZsq{}}\PY{p}{]}\PY{p}{]}\PY{p}{)}
         
         \PY{n}{f4} \PY{o}{=} \PY{n}{plt}\PY{o}{.}\PY{n}{gcf}\PY{p}{(}\PY{p}{)}
\end{Verbatim}


    \begin{center}
    \adjustimage{max size={0.9\linewidth}{0.9\paperheight}}{output_42_0.png}
    \end{center}
    { \hspace*{\fill} \\}
    
    \begin{Verbatim}[commandchars=\\\{\}]
{\color{incolor}In [{\color{incolor}23}]:} \PY{n}{scatter\PYZus{}matrix}\PY{p}{(}\PY{n}{df\PYZus{}combined}\PY{p}{[}\PY{p}{[}\PY{l+s+s1}{\PYZsq{}}\PY{l+s+s1}{volatile.acidity}\PY{l+s+s1}{\PYZsq{}}\PY{p}{,}\PY{l+s+s1}{\PYZsq{}}\PY{l+s+s1}{quality}\PY{l+s+s1}{\PYZsq{}}\PY{p}{]}\PY{p}{]}\PY{p}{)}
         
         \PY{n}{f4} \PY{o}{=} \PY{n}{plt}\PY{o}{.}\PY{n}{gcf}\PY{p}{(}\PY{p}{)}
\end{Verbatim}


    \begin{center}
    \adjustimage{max size={0.9\linewidth}{0.9\paperheight}}{output_43_0.png}
    \end{center}
    { \hspace*{\fill} \\}
    
    \begin{Verbatim}[commandchars=\\\{\}]
{\color{incolor}In [{\color{incolor}24}]:} \PY{n}{scatter\PYZus{}matrix}\PY{p}{(}\PY{n}{df\PYZus{}combined}\PY{p}{[}\PY{p}{[}\PY{l+s+s1}{\PYZsq{}}\PY{l+s+s1}{citric.acid}\PY{l+s+s1}{\PYZsq{}}\PY{p}{,}\PY{l+s+s1}{\PYZsq{}}\PY{l+s+s1}{quality}\PY{l+s+s1}{\PYZsq{}}\PY{p}{]}\PY{p}{]}\PY{p}{)}
         
         \PY{n}{f4} \PY{o}{=} \PY{n}{plt}\PY{o}{.}\PY{n}{gcf}\PY{p}{(}\PY{p}{)}
\end{Verbatim}


    \begin{center}
    \adjustimage{max size={0.9\linewidth}{0.9\paperheight}}{output_44_0.png}
    \end{center}
    { \hspace*{\fill} \\}
    
    \begin{Verbatim}[commandchars=\\\{\}]
{\color{incolor}In [{\color{incolor}25}]:} \PY{n}{scatter\PYZus{}matrix}\PY{p}{(}\PY{n}{df\PYZus{}combined}\PY{p}{[}\PY{p}{[}\PY{l+s+s1}{\PYZsq{}}\PY{l+s+s1}{residual.sugar}\PY{l+s+s1}{\PYZsq{}}\PY{p}{,}\PY{l+s+s1}{\PYZsq{}}\PY{l+s+s1}{quality}\PY{l+s+s1}{\PYZsq{}}\PY{p}{]}\PY{p}{]}\PY{p}{)}
         
         \PY{n}{f4} \PY{o}{=} \PY{n}{plt}\PY{o}{.}\PY{n}{gcf}\PY{p}{(}\PY{p}{)}
\end{Verbatim}


    \begin{center}
    \adjustimage{max size={0.9\linewidth}{0.9\paperheight}}{output_45_0.png}
    \end{center}
    { \hspace*{\fill} \\}
    
    \begin{Verbatim}[commandchars=\\\{\}]
{\color{incolor}In [{\color{incolor}26}]:} \PY{n}{scatter\PYZus{}matrix}\PY{p}{(}\PY{n}{df\PYZus{}combined}\PY{p}{[}\PY{p}{[}\PY{l+s+s1}{\PYZsq{}}\PY{l+s+s1}{chlorides}\PY{l+s+s1}{\PYZsq{}}\PY{p}{,}\PY{l+s+s1}{\PYZsq{}}\PY{l+s+s1}{quality}\PY{l+s+s1}{\PYZsq{}}\PY{p}{]}\PY{p}{]}\PY{p}{)}
         
         \PY{n}{f4} \PY{o}{=} \PY{n}{plt}\PY{o}{.}\PY{n}{gcf}\PY{p}{(}\PY{p}{)}
\end{Verbatim}


    \begin{center}
    \adjustimage{max size={0.9\linewidth}{0.9\paperheight}}{output_46_0.png}
    \end{center}
    { \hspace*{\fill} \\}
    
    \begin{Verbatim}[commandchars=\\\{\}]
{\color{incolor}In [{\color{incolor}27}]:} \PY{n}{scatter\PYZus{}matrix}\PY{p}{(}\PY{n}{df\PYZus{}combined}\PY{p}{[}\PY{p}{[}\PY{l+s+s1}{\PYZsq{}}\PY{l+s+s1}{free.sulfur.dioxide}\PY{l+s+s1}{\PYZsq{}}\PY{p}{,}\PY{l+s+s1}{\PYZsq{}}\PY{l+s+s1}{quality}\PY{l+s+s1}{\PYZsq{}}\PY{p}{]}\PY{p}{]}\PY{p}{)}
         
         \PY{n}{f4} \PY{o}{=} \PY{n}{plt}\PY{o}{.}\PY{n}{gcf}\PY{p}{(}\PY{p}{)}
\end{Verbatim}


    \begin{center}
    \adjustimage{max size={0.9\linewidth}{0.9\paperheight}}{output_47_0.png}
    \end{center}
    { \hspace*{\fill} \\}
    
    \begin{Verbatim}[commandchars=\\\{\}]
{\color{incolor}In [{\color{incolor}28}]:} \PY{n}{scatter\PYZus{}matrix}\PY{p}{(}\PY{n}{df\PYZus{}combined}\PY{p}{[}\PY{p}{[}\PY{l+s+s1}{\PYZsq{}}\PY{l+s+s1}{total.sulfur.dioxide}\PY{l+s+s1}{\PYZsq{}}\PY{p}{,}\PY{l+s+s1}{\PYZsq{}}\PY{l+s+s1}{quality}\PY{l+s+s1}{\PYZsq{}}\PY{p}{]}\PY{p}{]}\PY{p}{)}
         
         \PY{n}{f4} \PY{o}{=} \PY{n}{plt}\PY{o}{.}\PY{n}{gcf}\PY{p}{(}\PY{p}{)}
\end{Verbatim}


    \begin{center}
    \adjustimage{max size={0.9\linewidth}{0.9\paperheight}}{output_48_0.png}
    \end{center}
    { \hspace*{\fill} \\}
    
    \begin{Verbatim}[commandchars=\\\{\}]
{\color{incolor}In [{\color{incolor}29}]:} \PY{n}{scatter\PYZus{}matrix}\PY{p}{(}\PY{n}{df\PYZus{}combined}\PY{p}{[}\PY{p}{[}\PY{l+s+s1}{\PYZsq{}}\PY{l+s+s1}{density}\PY{l+s+s1}{\PYZsq{}}\PY{p}{,}\PY{l+s+s1}{\PYZsq{}}\PY{l+s+s1}{quality}\PY{l+s+s1}{\PYZsq{}}\PY{p}{]}\PY{p}{]}\PY{p}{)}
         
         \PY{n}{f4} \PY{o}{=} \PY{n}{plt}\PY{o}{.}\PY{n}{gcf}\PY{p}{(}\PY{p}{)}
\end{Verbatim}


    \begin{center}
    \adjustimage{max size={0.9\linewidth}{0.9\paperheight}}{output_49_0.png}
    \end{center}
    { \hspace*{\fill} \\}
    
    \begin{Verbatim}[commandchars=\\\{\}]
{\color{incolor}In [{\color{incolor}30}]:} \PY{n}{scatter\PYZus{}matrix}\PY{p}{(}\PY{n}{df\PYZus{}combined}\PY{p}{[}\PY{p}{[}\PY{l+s+s1}{\PYZsq{}}\PY{l+s+s1}{pH}\PY{l+s+s1}{\PYZsq{}}\PY{p}{,}\PY{l+s+s1}{\PYZsq{}}\PY{l+s+s1}{quality}\PY{l+s+s1}{\PYZsq{}}\PY{p}{]}\PY{p}{]}\PY{p}{)}
         
         \PY{n}{f4} \PY{o}{=} \PY{n}{plt}\PY{o}{.}\PY{n}{gcf}\PY{p}{(}\PY{p}{)}
\end{Verbatim}


    \begin{center}
    \adjustimage{max size={0.9\linewidth}{0.9\paperheight}}{output_50_0.png}
    \end{center}
    { \hspace*{\fill} \\}
    
    \begin{Verbatim}[commandchars=\\\{\}]
{\color{incolor}In [{\color{incolor}31}]:} \PY{n}{scatter\PYZus{}matrix}\PY{p}{(}\PY{n}{df\PYZus{}combined}\PY{p}{[}\PY{p}{[}\PY{l+s+s1}{\PYZsq{}}\PY{l+s+s1}{sulphates}\PY{l+s+s1}{\PYZsq{}}\PY{p}{,}\PY{l+s+s1}{\PYZsq{}}\PY{l+s+s1}{quality}\PY{l+s+s1}{\PYZsq{}}\PY{p}{]}\PY{p}{]}\PY{p}{)}
         
         \PY{n}{f4} \PY{o}{=} \PY{n}{plt}\PY{o}{.}\PY{n}{gcf}\PY{p}{(}\PY{p}{)}
\end{Verbatim}


    \begin{center}
    \adjustimage{max size={0.9\linewidth}{0.9\paperheight}}{output_51_0.png}
    \end{center}
    { \hspace*{\fill} \\}
    
    \begin{Verbatim}[commandchars=\\\{\}]
{\color{incolor}In [{\color{incolor}32}]:} \PY{n}{scatter\PYZus{}matrix}\PY{p}{(}\PY{n}{df\PYZus{}combined}\PY{p}{[}\PY{p}{[}\PY{l+s+s1}{\PYZsq{}}\PY{l+s+s1}{alcohol}\PY{l+s+s1}{\PYZsq{}}\PY{p}{,}\PY{l+s+s1}{\PYZsq{}}\PY{l+s+s1}{quality}\PY{l+s+s1}{\PYZsq{}}\PY{p}{]}\PY{p}{]}\PY{p}{)}
         
         \PY{n}{f4} \PY{o}{=} \PY{n}{plt}\PY{o}{.}\PY{n}{gcf}\PY{p}{(}\PY{p}{)}
\end{Verbatim}


    \begin{center}
    \adjustimage{max size={0.9\linewidth}{0.9\paperheight}}{output_52_0.png}
    \end{center}
    { \hspace*{\fill} \\}
    
    Since none of the feature seems to be highly correlated to the quality,
we want to combine them all and see if doing so would allow us to
predict the wine rating more accurately

    \begin{Verbatim}[commandchars=\\\{\}]
{\color{incolor}In [{\color{incolor}33}]:} \PY{n}{df\PYZus{}combined}\PY{o}{.}\PY{n}{shape}
\end{Verbatim}


\begin{Verbatim}[commandchars=\\\{\}]
{\color{outcolor}Out[{\color{outcolor}33}]:} (3198, 12)
\end{Verbatim}
            
    Similar to A5, we will split our dataset into training set (80\% of
total) and testing set (20\% of total).

    \begin{Verbatim}[commandchars=\\\{\}]
{\color{incolor}In [{\color{incolor}34}]:} \PY{n}{num\PYZus{}training} \PY{o}{=} \PY{n+nb}{int}\PY{p}{(}\PY{l+m+mi}{3198}\PY{o}{*}\PY{o}{.}\PY{l+m+mi}{8}\PY{p}{)}
         \PY{n}{num\PYZus{}training}
\end{Verbatim}


\begin{Verbatim}[commandchars=\\\{\}]
{\color{outcolor}Out[{\color{outcolor}34}]:} 2558
\end{Verbatim}
            
    \begin{Verbatim}[commandchars=\\\{\}]
{\color{incolor}In [{\color{incolor}35}]:} \PY{n}{df\PYZus{}training} \PY{o}{=} \PY{n}{df\PYZus{}combined}\PY{p}{[}\PY{p}{:}\PY{n}{num\PYZus{}training}\PY{p}{]}
         \PY{n}{df\PYZus{}testing} \PY{o}{=} \PY{n}{df\PYZus{}combined}\PY{p}{[}\PY{n}{num\PYZus{}training}\PY{p}{:}\PY{p}{]}
\end{Verbatim}


    The following show what out training set looks like if we ignore the
kind of the wine.

    \begin{Verbatim}[commandchars=\\\{\}]
{\color{incolor}In [{\color{incolor}36}]:} \PY{n}{df\PYZus{}training}\PY{p}{[}\PY{p}{:}\PY{l+m+mi}{5}\PY{p}{]}
\end{Verbatim}


\begin{Verbatim}[commandchars=\\\{\}]
{\color{outcolor}Out[{\color{outcolor}36}]:}    fixed.acidity  volatile.acidity  citric.acid  residual.sugar  chlorides  \textbackslash{}
         0            6.5              0.28         0.33            15.7      0.053   
         1            8.3              0.31         0.39             2.4      0.078   
         2            6.8              0.18         0.37             1.6      0.055   
         3            6.4              0.30         0.33             5.2      0.050   
         4            9.1              0.28         0.48             1.8      0.067   
         
            free.sulfur.dioxide  total.sulfur.dioxide  density    pH  sulphates  \textbackslash{}
         0                 51.0                 190.0  0.99780  3.22       0.51   
         1                 17.0                  43.0  0.99444  3.31       0.77   
         2                 47.0                 154.0  0.99340  3.08       0.45   
         3                 30.0                 137.0  0.99304  3.26       0.58   
         4                 26.0                  46.0  0.99670  3.32       1.04   
         
            alcohol  quality  
         0      9.7        6  
         1     12.5        7  
         2      9.1        5  
         3     11.1        5  
         4     10.6        6  
\end{Verbatim}
            
    Now, we will include some functions that can be useful later.

    \begin{Verbatim}[commandchars=\\\{\}]
{\color{incolor}In [{\color{incolor}37}]:} \PY{c+c1}{\PYZsh{}takes in a dataframe and a list of list of indices, }
         \PY{c+c1}{\PYZsh{}and return the predicted rate(quality) for each whine in the dataframe}
         \PY{c+c1}{\PYZsh{}based on the list of corresponding indices}
         \PY{k}{def} \PY{n+nf}{predict\PYZus{}rate}\PY{p}{(}\PY{n}{df}\PY{p}{,} \PY{n}{inds}\PY{p}{)}\PY{p}{:}
             \PY{n}{rate\PYZus{}predict} \PY{o}{=} \PY{p}{[}\PY{p}{]}
             \PY{k}{for} \PY{n}{i} \PY{o+ow}{in} \PY{n}{inds}\PY{p}{:}
                 \PY{n}{rateMap} \PY{o}{=} \PY{p}{\PYZob{}}\PY{p}{\PYZcb{}}
                 \PY{n}{rate\PYZus{}mode} \PY{o}{=} \PY{l+m+mi}{0}
                 \PY{k}{for} \PY{n}{j} \PY{o+ow}{in} \PY{n}{i}\PY{p}{:}
                     \PY{n}{curr} \PY{o}{=} \PY{n}{df}\PY{o}{.}\PY{n}{loc}\PY{p}{[}\PY{n}{j}\PY{p}{,} \PY{l+s+s1}{\PYZsq{}}\PY{l+s+s1}{quality}\PY{l+s+s1}{\PYZsq{}}\PY{p}{]}
                     \PY{k}{if} \PY{n}{curr} \PY{o+ow}{in} \PY{n}{rateMap}\PY{o}{.}\PY{n}{keys}\PY{p}{(}\PY{p}{)}\PY{p}{:}
                         \PY{n}{rateMap}\PY{p}{[}\PY{n}{curr}\PY{p}{]}\PY{o}{+}\PY{o}{=}\PY{l+m+mi}{1}
                     \PY{k}{else}\PY{p}{:} 
                         \PY{n}{rateMap}\PY{p}{[}\PY{n}{curr}\PY{p}{]} \PY{o}{=} \PY{l+m+mi}{1}
                     \PY{k}{if} \PY{n}{rateMap}\PY{p}{[}\PY{n}{curr}\PY{p}{]} \PY{o}{\PYZgt{}} \PY{n}{rate\PYZus{}mode}\PY{p}{:}
                         \PY{n}{rate\PYZus{}mode} \PY{o}{=} \PY{n}{curr}
                 \PY{n}{rate\PYZus{}predict}\PY{o}{.}\PY{n}{append}\PY{p}{(}\PY{n}{rate\PYZus{}mode}\PY{p}{)}
             \PY{k}{return} \PY{n}{rate\PYZus{}predict}
\end{Verbatim}


    \begin{Verbatim}[commandchars=\\\{\}]
{\color{incolor}In [{\color{incolor}38}]:} \PY{c+c1}{\PYZsh{}takes in a dataframe with an column containing the predicted wine quality}
         \PY{c+c1}{\PYZsh{}and returns the error rate}
         \PY{k}{def} \PY{n+nf}{validation}\PY{p}{(}\PY{n}{df\PYZus{}after\PYZus{}prediction}\PY{p}{)}\PY{p}{:}
             \PY{n}{error} \PY{o}{=} \PY{l+m+mi}{0}
             \PY{n}{num\PYZus{}row} \PY{o}{=} \PY{l+m+mi}{0}
             \PY{c+c1}{\PYZsh{} iterate through each row in df}
             \PY{k}{for} \PY{n}{idx}\PY{p}{,} \PY{n}{row} \PY{o+ow}{in} \PY{n}{df\PYZus{}after\PYZus{}prediction}\PY{o}{.}\PY{n}{iterrows}\PY{p}{(}\PY{p}{)}\PY{p}{:}
                 \PY{n}{num\PYZus{}row} \PY{o}{=} \PY{n}{num\PYZus{}row}\PY{o}{+}\PY{l+m+mi}{1}
                 \PY{k}{if} \PY{o+ow}{not} \PY{n}{row}\PY{p}{[}\PY{l+s+s1}{\PYZsq{}}\PY{l+s+s1}{quality}\PY{l+s+s1}{\PYZsq{}}\PY{p}{]} \PY{o}{==} \PY{n}{row}\PY{p}{[}\PY{l+s+s1}{\PYZsq{}}\PY{l+s+s1}{predicted quality}\PY{l+s+s1}{\PYZsq{}}\PY{p}{]}\PY{p}{:}
                     \PY{n}{error} \PY{o}{=} \PY{n}{error}\PY{o}{+}\PY{l+m+mi}{1}
             \PY{n}{errorPercent} \PY{o}{=} \PY{n}{error}\PY{o}{/}\PY{n}{num\PYZus{}row}
             \PY{c+c1}{\PYZsh{}print(errorPercent)}
             \PY{k}{return} \PY{n}{errorPercent}
\end{Verbatim}


    \begin{Verbatim}[commandchars=\\\{\}]
{\color{incolor}In [{\color{incolor}39}]:} \PY{c+c1}{\PYZsh{}takes in a dataframe}
         \PY{c+c1}{\PYZsh{}and return a list of rows in the dataframe without the quality column}
         \PY{k}{def} \PY{n+nf}{get\PYZus{}row\PYZus{}list}\PY{p}{(}\PY{n}{df}\PY{p}{)}\PY{p}{:}
             \PY{n}{rows\PYZus{}list}\PY{o}{=}\PY{p}{[}\PY{p}{]}
             \PY{k}{for} \PY{n}{idx}\PY{p}{,} \PY{n}{row} \PY{o+ow}{in} \PY{n}{df}\PY{o}{.}\PY{n}{iterrows}\PY{p}{(}\PY{p}{)}\PY{p}{:}
                 \PY{n}{curr\PYZus{}row} \PY{o}{=} \PY{p}{[}\PY{n}{row}\PY{p}{[}\PY{l+s+s1}{\PYZsq{}}\PY{l+s+s1}{fixed.acidity}\PY{l+s+s1}{\PYZsq{}}\PY{p}{]}\PY{p}{,} \PY{n}{row}\PY{p}{[}\PY{l+s+s1}{\PYZsq{}}\PY{l+s+s1}{volatile.acidity}\PY{l+s+s1}{\PYZsq{}}\PY{p}{]}\PY{p}{,} \PY{n}{row}\PY{p}{[}\PY{l+s+s1}{\PYZsq{}}\PY{l+s+s1}{citric.acid}\PY{l+s+s1}{\PYZsq{}}\PY{p}{]}\PY{p}{,} 
                             \PY{n}{row}\PY{p}{[}\PY{l+s+s1}{\PYZsq{}}\PY{l+s+s1}{residual.sugar}\PY{l+s+s1}{\PYZsq{}}\PY{p}{]}\PY{p}{,} \PY{n}{row}\PY{p}{[}\PY{l+s+s1}{\PYZsq{}}\PY{l+s+s1}{chlorides}\PY{l+s+s1}{\PYZsq{}}\PY{p}{]}\PY{p}{,} \PY{n}{row}\PY{p}{[}\PY{l+s+s1}{\PYZsq{}}\PY{l+s+s1}{free.sulfur.dioxide}\PY{l+s+s1}{\PYZsq{}}\PY{p}{]}\PY{p}{,} 
                             \PY{n}{row}\PY{p}{[}\PY{l+s+s1}{\PYZsq{}}\PY{l+s+s1}{total.sulfur.dioxide}\PY{l+s+s1}{\PYZsq{}}\PY{p}{]}\PY{p}{,} \PY{n}{row}\PY{p}{[}\PY{l+s+s1}{\PYZsq{}}\PY{l+s+s1}{density}\PY{l+s+s1}{\PYZsq{}}\PY{p}{]}\PY{p}{,} \PY{n}{row}\PY{p}{[}\PY{l+s+s1}{\PYZsq{}}\PY{l+s+s1}{pH}\PY{l+s+s1}{\PYZsq{}}\PY{p}{]}\PY{p}{,} \PY{n}{row}\PY{p}{[}\PY{l+s+s1}{\PYZsq{}}\PY{l+s+s1}{sulphates}\PY{l+s+s1}{\PYZsq{}}\PY{p}{]}\PY{p}{,}
                             \PY{n}{row}\PY{p}{[}\PY{l+s+s1}{\PYZsq{}}\PY{l+s+s1}{alcohol}\PY{l+s+s1}{\PYZsq{}}\PY{p}{]}\PY{p}{]}
                 \PY{n}{rows\PYZus{}list}\PY{o}{.}\PY{n}{append}\PY{p}{(}\PY{n}{curr\PYZus{}row}\PY{p}{)} 
             \PY{k}{return} \PY{n}{rows\PYZus{}list}
\end{Verbatim}


    \begin{Verbatim}[commandchars=\\\{\}]
{\color{incolor}In [{\color{incolor}40}]:} \PY{c+c1}{\PYZsh{}takes in the training dataframe, the testing dataframe, the k value, the testing dataframe\PYZsq{}s row list, and a kd\PYZhy{}tree, }
         \PY{c+c1}{\PYZsh{}and returns the result of validation}
         \PY{k}{def} \PY{n+nf}{getErrorPercent}\PY{p}{(}\PY{n}{df\PYZus{}training}\PY{p}{,} \PY{n}{df\PYZus{}testing}\PY{p}{,} \PY{n}{k\PYZus{}value}\PY{p}{,} \PY{n}{rows\PYZus{}list}\PY{p}{,} \PY{n}{tree}\PY{p}{)}\PY{p}{:}
             \PY{n}{inds} \PY{o}{=} \PY{n}{tree}\PY{o}{.}\PY{n}{query}\PY{p}{(}\PY{n}{rows\PYZus{}list}\PY{p}{,} \PY{n}{k}\PY{o}{=}\PY{n}{k\PYZus{}value}\PY{p}{,} \PY{n}{return\PYZus{}distance} \PY{o}{=} \PY{k+kc}{False}\PY{p}{)}
             \PY{n}{df\PYZus{}col\PYZus{}pred} \PY{o}{=} \PY{n}{predict\PYZus{}rate}\PY{p}{(}\PY{n}{df\PYZus{}training}\PY{p}{,} \PY{n}{inds}\PY{p}{)}
             \PY{n}{df\PYZus{}testing}\PY{p}{[}\PY{l+s+s1}{\PYZsq{}}\PY{l+s+s1}{predicted quality}\PY{l+s+s1}{\PYZsq{}}\PY{p}{]} \PY{o}{=} \PY{n}{df\PYZus{}col\PYZus{}pred}
             \PY{k}{return} \PY{n}{validation}\PY{p}{(}\PY{n}{df\PYZus{}testing}\PY{p}{)}
\end{Verbatim}


    We want to use the KD-tree in the sklearn library to predict the quality
of a wine given its list of features(ingredients). A K-Dimensional tree
is a tree data structure for organizing sample points in a k dimensional
space. In out project, we have 11 dimensions as there are 11 features
for each wine. A KDT will store our sample points (the rows of wine)
according to all their features. Therefore, by querying the closest
"neighbors" of a wine from the KDT, we get a list of other wines in the
KDT positioned the closest to the query point. Since we have information
about those wines' quality, we can then predict the query wine's quality
by taking the quality with the highest occurence among the nearest
neighbors.

    Now, we will build the KDT using the training dataset.

    \begin{Verbatim}[commandchars=\\\{\}]
{\color{incolor}In [{\color{incolor}41}]:} \PY{c+c1}{\PYZsh{}list of rows of the training dataframe}
         \PY{n}{rows\PYZus{}list\PYZus{}training} \PY{o}{=} \PY{n}{get\PYZus{}row\PYZus{}list}\PY{p}{(}\PY{n}{df\PYZus{}training}\PY{p}{)}
         
         \PY{c+c1}{\PYZsh{}the KD Tree that stores all the wines in the training set}
         \PY{n}{tree} \PY{o}{=} \PY{n}{KDTree}\PY{p}{(}\PY{n}{rows\PYZus{}list\PYZus{}training}\PY{p}{)}
\end{Verbatim}


    After building the KD tree, we want to know how many numbers of
neighbors we should query from it in order to get the most accurate
prediction. Out initial guess is 5, but we want to actually go through
more possible values to make sure we choose a best k value.

    \begin{Verbatim}[commandchars=\\\{\}]
{\color{incolor}In [{\color{incolor}42}]:} \PY{c+c1}{\PYZsh{}going through possible k values [1,40]}
         \PY{n}{k\PYZus{}val} \PY{o}{=} \PY{n}{np}\PY{o}{.}\PY{n}{arange}\PY{p}{(}\PY{l+m+mi}{1}\PY{p}{,} \PY{l+m+mi}{41}\PY{p}{,} \PY{l+m+mi}{1}\PY{p}{)}
         \PY{n}{err\PYZus{}val}\PY{o}{=}\PY{p}{[}\PY{p}{]}
         \PY{k}{for} \PY{n}{i} \PY{o+ow}{in} \PY{n}{k\PYZus{}val}\PY{p}{:}
             \PY{n}{df\PYZus{}training\PYZus{}cpy} \PY{o}{=} \PY{n}{df\PYZus{}training}\PY{o}{.}\PY{n}{copy}\PY{p}{(}\PY{p}{)}
             \PY{n}{err\PYZus{}val}\PY{o}{.}\PY{n}{append}\PY{p}{(}\PY{n}{getErrorPercent}\PY{p}{(}\PY{n}{df\PYZus{}training}\PY{p}{,} \PY{n}{df\PYZus{}training\PYZus{}cpy}\PY{p}{,} \PY{n}{i} \PY{p}{,}\PY{n}{rows\PYZus{}list\PYZus{}training}\PY{p}{,} \PY{n}{tree}\PY{p}{)}\PY{p}{)}
         \PY{n}{df\PYZus{}err} \PY{o}{=} \PY{n}{pd}\PY{o}{.}\PY{n}{DataFrame}\PY{p}{(}\PY{p}{)}
         \PY{n}{df\PYZus{}err}\PY{p}{[}\PY{l+s+s1}{\PYZsq{}}\PY{l+s+s1}{k value}\PY{l+s+s1}{\PYZsq{}}\PY{p}{]} \PY{o}{=} \PY{n}{k\PYZus{}val}
         \PY{n}{df\PYZus{}err}\PY{p}{[}\PY{l+s+s1}{\PYZsq{}}\PY{l+s+s1}{error}\PY{l+s+s1}{\PYZsq{}}\PY{p}{]} \PY{o}{=} \PY{n}{err\PYZus{}val}
\end{Verbatim}


    \begin{Verbatim}[commandchars=\\\{\}]
{\color{incolor}In [{\color{incolor}43}]:} \PY{n}{df\PYZus{}err}\PY{p}{[}\PY{p}{:}\PY{l+m+mi}{10}\PY{p}{]}
\end{Verbatim}


\begin{Verbatim}[commandchars=\\\{\}]
{\color{outcolor}Out[{\color{outcolor}43}]:}    k value     error
         0        1  0.000000
         1        2  0.000000
         2        3  0.000000
         3        4  0.000000
         4        5  0.000000
         5        6  0.003518
         6        7  0.013292
         7        8  0.034011
         8        9  0.061376
         9       10  0.097733
\end{Verbatim}
            
    \begin{Verbatim}[commandchars=\\\{\}]
{\color{incolor}In [{\color{incolor}44}]:} \PY{n}{df\PYZus{}err}
         \PY{n}{df\PYZus{}err}\PY{o}{.}\PY{n}{plot}\PY{o}{.}\PY{n}{scatter}\PY{p}{(}\PY{n}{x}\PY{o}{=}\PY{l+s+s1}{\PYZsq{}}\PY{l+s+s1}{k value}\PY{l+s+s1}{\PYZsq{}}\PY{p}{,} \PY{n}{y}\PY{o}{=}\PY{l+s+s1}{\PYZsq{}}\PY{l+s+s1}{error}\PY{l+s+s1}{\PYZsq{}}\PY{p}{)}
\end{Verbatim}


\begin{Verbatim}[commandchars=\\\{\}]
{\color{outcolor}Out[{\color{outcolor}44}]:} <matplotlib.axes.\_subplots.AxesSubplot at 0x1a0d745320>
\end{Verbatim}
            
    \begin{center}
    \adjustimage{max size={0.9\linewidth}{0.9\paperheight}}{output_71_1.png}
    \end{center}
    { \hspace*{\fill} \\}
    
    After checking through a set of k value candidates, we decide that k=5
would be the best choice. Because it gives us a very low error. Although
k values less than 5 seem to have even lower error rates, there are a
few reasons why we did not choose them:

\begin{enumerate}
\def\labelenumi{\arabic{enumi})}
\item
  We do not want to choose an even k value becuase we will then need to
  worry about having a tie.
\item
  k=1 and k=3 are odd numbers, but still not nice choices as they are
  too small.
\end{enumerate}

    We will now query 5 nearest neighbors for each wine instance in our
training data set, predict the quality, and add to the dataframe.

    \begin{Verbatim}[commandchars=\\\{\}]
{\color{incolor}In [{\color{incolor}45}]:} \PY{c+c1}{\PYZsh{}the list of rows of the training dataset}
         \PY{n}{rows\PYZus{}list\PYZus{}training} \PY{o}{=} \PY{n}{get\PYZus{}row\PYZus{}list}\PY{p}{(}\PY{n}{df\PYZus{}training}\PY{p}{)}
         
         \PY{c+c1}{\PYZsh{}the list of lists of nearest neighbor indices}
         \PY{n}{inds\PYZus{}k\PYZus{}5\PYZus{}training} \PY{o}{=} \PY{n}{tree}\PY{o}{.}\PY{n}{query}\PY{p}{(}\PY{n}{rows\PYZus{}list\PYZus{}training}\PY{p}{,} \PY{n}{k}\PY{o}{=}\PY{l+m+mi}{5}\PY{p}{,} \PY{n}{return\PYZus{}distance} \PY{o}{=} \PY{k+kc}{False}\PY{p}{)}
         
         \PY{c+c1}{\PYZsh{}the list of predicted quality of each wine}
         \PY{n}{df\PYZus{}training}\PY{p}{[}\PY{l+s+s1}{\PYZsq{}}\PY{l+s+s1}{predicted quality}\PY{l+s+s1}{\PYZsq{}}\PY{p}{]} \PY{o}{=} \PY{n}{predict\PYZus{}rate}\PY{p}{(}\PY{n}{df\PYZus{}training}\PY{p}{,} \PY{n}{inds\PYZus{}k\PYZus{}5\PYZus{}training}\PY{p}{)}
\end{Verbatim}


    \begin{Verbatim}[commandchars=\\\{\}]
/Users/aitinghsieh/anaconda3/lib/python3.6/site-packages/ipykernel\_launcher.py:8: SettingWithCopyWarning: 
A value is trying to be set on a copy of a slice from a DataFrame.
Try using .loc[row\_indexer,col\_indexer] = value instead

See the caveats in the documentation: http://pandas.pydata.org/pandas-docs/stable/indexing.html\#indexing-view-versus-copy
  

    \end{Verbatim}

    \begin{Verbatim}[commandchars=\\\{\}]
{\color{incolor}In [{\color{incolor}46}]:} \PY{n}{df\PYZus{}training}\PY{p}{[}\PY{p}{:}\PY{l+m+mi}{5}\PY{p}{]}
\end{Verbatim}


\begin{Verbatim}[commandchars=\\\{\}]
{\color{outcolor}Out[{\color{outcolor}46}]:}    fixed.acidity  volatile.acidity  citric.acid  residual.sugar  chlorides  \textbackslash{}
         0            6.5              0.28         0.33            15.7      0.053   
         1            8.3              0.31         0.39             2.4      0.078   
         2            6.8              0.18         0.37             1.6      0.055   
         3            6.4              0.30         0.33             5.2      0.050   
         4            9.1              0.28         0.48             1.8      0.067   
         
            free.sulfur.dioxide  total.sulfur.dioxide  density    pH  sulphates  \textbackslash{}
         0                 51.0                 190.0  0.99780  3.22       0.51   
         1                 17.0                  43.0  0.99444  3.31       0.77   
         2                 47.0                 154.0  0.99340  3.08       0.45   
         3                 30.0                 137.0  0.99304  3.26       0.58   
         4                 26.0                  46.0  0.99670  3.32       1.04   
         
            alcohol  quality  predicted quality  
         0      9.7        6                  6  
         1     12.5        7                  7  
         2      9.1        5                  5  
         3     11.1        5                  5  
         4     10.6        6                  6  
\end{Verbatim}
            
    We will compute the error percentage of our prediction.

    \begin{Verbatim}[commandchars=\\\{\}]
{\color{incolor}In [{\color{incolor}47}]:} \PY{n}{error\PYZus{}training} \PY{o}{=} \PY{n}{getErrorPercent}\PY{p}{(}\PY{n}{df\PYZus{}training}\PY{p}{,} \PY{n}{df\PYZus{}training}\PY{p}{,} \PY{l+m+mi}{5}\PY{p}{,} \PY{n}{rows\PYZus{}list\PYZus{}training}\PY{p}{,}\PY{n}{tree}\PY{p}{)}
         \PY{n}{error\PYZus{}training}
\end{Verbatim}


    \begin{Verbatim}[commandchars=\\\{\}]
/Users/aitinghsieh/anaconda3/lib/python3.6/site-packages/ipykernel\_launcher.py:6: SettingWithCopyWarning: 
A value is trying to be set on a copy of a slice from a DataFrame.
Try using .loc[row\_indexer,col\_indexer] = value instead

See the caveats in the documentation: http://pandas.pydata.org/pandas-docs/stable/indexing.html\#indexing-view-versus-copy
  

    \end{Verbatim}

\begin{Verbatim}[commandchars=\\\{\}]
{\color{outcolor}Out[{\color{outcolor}47}]:} 0.0
\end{Verbatim}
            
    Now that our model is ready, we will go ahead and start predicting the
quality of wines in the testing dataset.

    \begin{Verbatim}[commandchars=\\\{\}]
{\color{incolor}In [{\color{incolor}48}]:} \PY{c+c1}{\PYZsh{}the list of rows in the testing set}
         \PY{n}{rows\PYZus{}list\PYZus{}testing} \PY{o}{=} \PY{n}{get\PYZus{}row\PYZus{}list}\PY{p}{(}\PY{n}{df\PYZus{}testing}\PY{p}{)}
         
         \PY{c+c1}{\PYZsh{}the list of lists of nearest neighbor indices}
         \PY{n}{inds\PYZus{}k\PYZus{}5\PYZus{}testing} \PY{o}{=} \PY{n}{tree}\PY{o}{.}\PY{n}{query}\PY{p}{(}\PY{n}{rows\PYZus{}list\PYZus{}testing}\PY{p}{,} \PY{n}{k}\PY{o}{=}\PY{l+m+mi}{5}\PY{p}{,} \PY{n}{return\PYZus{}distance} \PY{o}{=} \PY{k+kc}{False}\PY{p}{)}
         
         \PY{c+c1}{\PYZsh{}the list of predicted quality of each wine}
         \PY{n}{df\PYZus{}testing}\PY{p}{[}\PY{l+s+s1}{\PYZsq{}}\PY{l+s+s1}{predicted quality}\PY{l+s+s1}{\PYZsq{}}\PY{p}{]} \PY{o}{=} \PY{n}{predict\PYZus{}rate}\PY{p}{(}\PY{n}{df\PYZus{}training}\PY{p}{,} \PY{n}{inds\PYZus{}k\PYZus{}5\PYZus{}testing}\PY{p}{)}
\end{Verbatim}


    \begin{Verbatim}[commandchars=\\\{\}]
/Users/aitinghsieh/anaconda3/lib/python3.6/site-packages/ipykernel\_launcher.py:8: SettingWithCopyWarning: 
A value is trying to be set on a copy of a slice from a DataFrame.
Try using .loc[row\_indexer,col\_indexer] = value instead

See the caveats in the documentation: http://pandas.pydata.org/pandas-docs/stable/indexing.html\#indexing-view-versus-copy
  

    \end{Verbatim}

    \begin{Verbatim}[commandchars=\\\{\}]
{\color{incolor}In [{\color{incolor}49}]:} \PY{c+c1}{\PYZsh{} err\PYZus{}testing = getErrorPercent(df\PYZus{}training, df\PYZus{}testing, 5, rows\PYZus{}list\PYZus{}testing)}
         \PY{n}{err\PYZus{}testing} \PY{o}{=} \PY{n}{validation}\PY{p}{(}\PY{n}{df\PYZus{}testing}\PY{p}{)}
         \PY{n}{err\PYZus{}testing}
\end{Verbatim}


\begin{Verbatim}[commandchars=\\\{\}]
{\color{outcolor}Out[{\color{outcolor}49}]:} 0.4609375
\end{Verbatim}
            
    \section{Data Analysis / Results}\label{data-analysis-results}

    It turned out that the error percentage is pretty big, oppposite from
what we were expecting. Therefore, now we want to see how in accurate
our prediction is.

    \begin{Verbatim}[commandchars=\\\{\}]
{\color{incolor}In [{\color{incolor}50}]:} \PY{n}{df\PYZus{}testing}\PY{p}{[}\PY{p}{:}\PY{l+m+mi}{10}\PY{p}{]}
\end{Verbatim}


\begin{Verbatim}[commandchars=\\\{\}]
{\color{outcolor}Out[{\color{outcolor}50}]:}       fixed.acidity  volatile.acidity  citric.acid  residual.sugar  chlorides  \textbackslash{}
         2558            7.9              0.54         0.34            2.50      0.076   
         2559            7.3              0.30         0.34            2.70      0.044   
         2560            6.7              0.46         0.24            1.70      0.077   
         2561            6.7              0.26         0.26            4.10      0.073   
         2562            7.4              0.19         0.30            1.40      0.057   
         2563            5.7              0.22         0.22           16.65      0.044   
         2564           12.3              0.27         0.49            3.10      0.079   
         2565            7.3              0.91         0.10            1.80      0.074   
         2566            9.3              0.40         0.49            2.50      0.085   
         2567            6.1              0.20         0.40            1.90      0.028   
         
               free.sulfur.dioxide  total.sulfur.dioxide  density    pH  sulphates  \textbackslash{}
         2558                  8.0                  17.0  0.99235  3.20       0.72   
         2559                 34.0                 108.0  0.99105  3.36       0.53   
         2560                 18.0                  34.0  0.99480  3.39       0.60   
         2561                 36.0                 202.0  0.99560  3.30       0.67   
         2562                 33.0                 135.0  0.99300  3.12       0.50   
         2563                 39.0                 110.0  0.99855  3.24       0.48   
         2564                 28.0                  46.0  0.99930  3.20       0.80   
         2565                 20.0                  56.0  0.99672  3.35       0.56   
         2566                 38.0                 142.0  0.99780  3.22       0.55   
         2567                 32.0                 138.0  0.99140  3.26       0.72   
         
               alcohol  quality  predicted quality  
         2558     13.1        8                  6  
         2559     12.8        8                  6  
         2560     10.6        6                  6  
         2561      9.5        5                  4  
         2562      9.6        6                  7  
         2563      9.0        6                  6  
         2564     10.2        6                  4  
         2565      9.2        5                  5  
         2566      9.4        5                  5  
         2567     11.7        5                  6  
\end{Verbatim}
            
    It looks like some wines have the same predicted and actual quality
while some others don't. We will need to do some more work to see how
our algorithm performs.

    \begin{Verbatim}[commandchars=\\\{\}]
{\color{incolor}In [{\color{incolor}51}]:} \PY{k+kn}{import} \PY{n+nn}{matplotlib}\PY{n+nn}{.}\PY{n+nn}{pyplot} \PY{k}{as} \PY{n+nn}{plt}
         
         \PY{n}{fig} \PY{o}{=} \PY{n}{plt}\PY{o}{.}\PY{n}{figure}\PY{p}{(}\PY{p}{)}
         \PY{n}{fig}\PY{o}{.}\PY{n}{set\PYZus{}size\PYZus{}inches}\PY{p}{(}\PY{l+m+mf}{18.5}\PY{p}{,} \PY{l+m+mf}{10.5}\PY{p}{)}
         \PY{n}{ax1} \PY{o}{=} \PY{n}{fig}\PY{o}{.}\PY{n}{add\PYZus{}subplot}\PY{p}{(}\PY{l+m+mi}{111}\PY{p}{)}
         
         \PY{n}{ax1}\PY{o}{.}\PY{n}{scatter}\PY{p}{(}\PY{n}{df\PYZus{}testing}\PY{o}{.}\PY{n}{index}\PY{p}{,} \PY{n}{df\PYZus{}testing}\PY{p}{[}\PY{l+s+s1}{\PYZsq{}}\PY{l+s+s1}{predicted quality}\PY{l+s+s1}{\PYZsq{}}\PY{p}{]}\PY{p}{,} \PY{n}{s}\PY{o}{=}\PY{l+m+mi}{10}\PY{p}{,} \PY{n}{c}\PY{o}{=}\PY{l+s+s1}{\PYZsq{}}\PY{l+s+s1}{b}\PY{l+s+s1}{\PYZsq{}}\PY{p}{,} \PY{n}{marker}\PY{o}{=}\PY{l+s+s2}{\PYZdq{}}\PY{l+s+s2}{s}\PY{l+s+s2}{\PYZdq{}}\PY{p}{,} \PY{n}{label}\PY{o}{=}\PY{l+s+s1}{\PYZsq{}}\PY{l+s+s1}{predicted quality}\PY{l+s+s1}{\PYZsq{}}\PY{p}{)}
         \PY{n}{ax1}\PY{o}{.}\PY{n}{scatter}\PY{p}{(}\PY{n}{df\PYZus{}testing}\PY{o}{.}\PY{n}{index}\PY{p}{,} \PY{n}{df\PYZus{}testing}\PY{p}{[}\PY{l+s+s1}{\PYZsq{}}\PY{l+s+s1}{quality}\PY{l+s+s1}{\PYZsq{}}\PY{p}{]}\PY{p}{,} \PY{n}{s}\PY{o}{=}\PY{l+m+mi}{10}\PY{p}{,} \PY{n}{c}\PY{o}{=}\PY{l+s+s1}{\PYZsq{}}\PY{l+s+s1}{r}\PY{l+s+s1}{\PYZsq{}}\PY{p}{,} \PY{n}{marker}\PY{o}{=}\PY{l+s+s2}{\PYZdq{}}\PY{l+s+s2}{o}\PY{l+s+s2}{\PYZdq{}}\PY{p}{,} \PY{n}{label}\PY{o}{=}\PY{l+s+s1}{\PYZsq{}}\PY{l+s+s1}{real quality}\PY{l+s+s1}{\PYZsq{}}\PY{p}{)}
         \PY{n}{plt}\PY{o}{.}\PY{n}{legend}\PY{p}{(}\PY{n}{loc}\PY{o}{=}\PY{l+s+s1}{\PYZsq{}}\PY{l+s+s1}{upper left}\PY{l+s+s1}{\PYZsq{}}\PY{p}{)}
         \PY{n}{plt}\PY{o}{.}\PY{n}{show}\PY{p}{(}\PY{p}{)}
\end{Verbatim}


    \begin{center}
    \adjustimage{max size={0.9\linewidth}{0.9\paperheight}}{output_85_0.png}
    \end{center}
    { \hspace*{\fill} \\}
    
    The scatter plot above does not look so nice and we cannot really tell
much about our prediction by looking at it. Thus, we will calculate the
root mean sqaure error we got and store the value in our dataframe.

    \begin{Verbatim}[commandchars=\\\{\}]
{\color{incolor}In [{\color{incolor}52}]:} \PY{c+c1}{\PYZsh{}takes in two values and calculautes their root mean square difference}
         \PY{c+c1}{\PYZsh{}and then returns that result}
         \PY{k}{def} \PY{n+nf}{rmse}\PY{p}{(}\PY{n}{pred}\PY{p}{,} \PY{n}{tar}\PY{p}{)}\PY{p}{:}
             \PY{n}{diff} \PY{o}{=} \PY{n}{pred} \PY{o}{\PYZhy{}} \PY{n}{tar}                       
             \PY{n}{diff\PYZus{}sq} \PY{o}{=} \PY{n}{diff} \PY{o}{*}\PY{o}{*} \PY{l+m+mi}{2}                    
             \PY{n}{mean\PYZus{}diff\PYZus{}sq} \PY{o}{=} \PY{n}{diff\PYZus{}sq}\PY{o}{.}\PY{n}{mean}\PY{p}{(}\PY{p}{)}  
             \PY{n}{rmse\PYZus{}val} \PY{o}{=} \PY{n}{np}\PY{o}{.}\PY{n}{sqrt}\PY{p}{(}\PY{n}{mean\PYZus{}diff\PYZus{}sq}\PY{p}{)}          
             \PY{k}{return} \PY{n}{rmse\PYZus{}val}\PY{o}{*}\PY{l+m+mf}{1.0}
\end{Verbatim}


    \begin{Verbatim}[commandchars=\\\{\}]
{\color{incolor}In [{\color{incolor}53}]:} \PY{n}{df\PYZus{}testing}\PY{p}{[}\PY{l+s+s1}{\PYZsq{}}\PY{l+s+s1}{rms}\PY{l+s+s1}{\PYZsq{}}\PY{p}{]}\PY{o}{=}\PY{l+m+mf}{0.0}
         \PY{k}{for} \PY{n}{i} \PY{o+ow}{in} \PY{n}{df\PYZus{}testing}\PY{o}{.}\PY{n}{index}\PY{p}{:}
                 \PY{n}{df\PYZus{}testing}\PY{o}{.}\PY{n}{at}\PY{p}{[}\PY{n}{i}\PY{p}{,}\PY{l+s+s1}{\PYZsq{}}\PY{l+s+s1}{rms}\PY{l+s+s1}{\PYZsq{}}\PY{p}{]} \PY{o}{=} \PY{n}{rmse}\PY{p}{(}\PY{n}{df\PYZus{}testing}\PY{o}{.}\PY{n}{at}\PY{p}{[}\PY{n}{i}\PY{p}{,}\PY{l+s+s1}{\PYZsq{}}\PY{l+s+s1}{predicted quality}\PY{l+s+s1}{\PYZsq{}}\PY{p}{]}\PY{p}{,} \PY{n}{df\PYZus{}testing}\PY{o}{.}\PY{n}{at}\PY{p}{[}\PY{n}{i}\PY{p}{,}\PY{l+s+s1}{\PYZsq{}}\PY{l+s+s1}{quality}\PY{l+s+s1}{\PYZsq{}}\PY{p}{]}\PY{p}{)}
\end{Verbatim}


    \begin{Verbatim}[commandchars=\\\{\}]
/Users/aitinghsieh/anaconda3/lib/python3.6/site-packages/ipykernel\_launcher.py:1: SettingWithCopyWarning: 
A value is trying to be set on a copy of a slice from a DataFrame.
Try using .loc[row\_indexer,col\_indexer] = value instead

See the caveats in the documentation: http://pandas.pydata.org/pandas-docs/stable/indexing.html\#indexing-view-versus-copy
  """Entry point for launching an IPython kernel.

    \end{Verbatim}

    \begin{Verbatim}[commandchars=\\\{\}]
{\color{incolor}In [{\color{incolor}54}]:} \PY{n}{df\PYZus{}testing}\PY{p}{[}\PY{p}{:}\PY{l+m+mi}{10}\PY{p}{]}
\end{Verbatim}


\begin{Verbatim}[commandchars=\\\{\}]
{\color{outcolor}Out[{\color{outcolor}54}]:}       fixed.acidity  volatile.acidity  citric.acid  residual.sugar  chlorides  \textbackslash{}
         2558            7.9              0.54         0.34            2.50      0.076   
         2559            7.3              0.30         0.34            2.70      0.044   
         2560            6.7              0.46         0.24            1.70      0.077   
         2561            6.7              0.26         0.26            4.10      0.073   
         2562            7.4              0.19         0.30            1.40      0.057   
         2563            5.7              0.22         0.22           16.65      0.044   
         2564           12.3              0.27         0.49            3.10      0.079   
         2565            7.3              0.91         0.10            1.80      0.074   
         2566            9.3              0.40         0.49            2.50      0.085   
         2567            6.1              0.20         0.40            1.90      0.028   
         
               free.sulfur.dioxide  total.sulfur.dioxide  density    pH  sulphates  \textbackslash{}
         2558                  8.0                  17.0  0.99235  3.20       0.72   
         2559                 34.0                 108.0  0.99105  3.36       0.53   
         2560                 18.0                  34.0  0.99480  3.39       0.60   
         2561                 36.0                 202.0  0.99560  3.30       0.67   
         2562                 33.0                 135.0  0.99300  3.12       0.50   
         2563                 39.0                 110.0  0.99855  3.24       0.48   
         2564                 28.0                  46.0  0.99930  3.20       0.80   
         2565                 20.0                  56.0  0.99672  3.35       0.56   
         2566                 38.0                 142.0  0.99780  3.22       0.55   
         2567                 32.0                 138.0  0.99140  3.26       0.72   
         
               alcohol  quality  predicted quality  rms  
         2558     13.1        8                  6  2.0  
         2559     12.8        8                  6  2.0  
         2560     10.6        6                  6  0.0  
         2561      9.5        5                  4  1.0  
         2562      9.6        6                  7  1.0  
         2563      9.0        6                  6  0.0  
         2564     10.2        6                  4  2.0  
         2565      9.2        5                  5  0.0  
         2566      9.4        5                  5  0.0  
         2567     11.7        5                  6  1.0  
\end{Verbatim}
            
    \begin{Verbatim}[commandchars=\\\{\}]
{\color{incolor}In [{\color{incolor}55}]:} \PY{n}{df\PYZus{}testing}\PY{p}{[}\PY{l+s+s1}{\PYZsq{}}\PY{l+s+s1}{rms}\PY{l+s+s1}{\PYZsq{}}\PY{p}{]}\PY{o}{.}\PY{n}{plot}\PY{o}{.}\PY{n}{hist}\PY{p}{(}\PY{p}{)}
\end{Verbatim}


\begin{Verbatim}[commandchars=\\\{\}]
{\color{outcolor}Out[{\color{outcolor}55}]:} <matplotlib.axes.\_subplots.AxesSubplot at 0x1a0d91fc18>
\end{Verbatim}
            
    \begin{center}
    \adjustimage{max size={0.9\linewidth}{0.9\paperheight}}{output_90_1.png}
    \end{center}
    { \hspace*{\fill} \\}
    
    We can see that the error we get is actually very small. The predicted
quality is in the most case actual quality ± 1, with a few ones having a
more inaccurate result. So after all, our prediction is still not bad.

    \section{Ethics and Privacy}\label{ethics-and-privacy}

    In our ventures in trying to determine what predicts wine quality, it is
important to consider who and what may be negatively impacted by the
results we obtain. Our results may reveal that certain qualities in wine
are more favorable than others, affecting the business of companies that
may be known to have more or less of a favorable quality. For example,
if it is found that a wine with a higher sugar content is less
favorable, companies and vineyards known for sweeter wines may find
their businesses negatively impacted. Additionally, if a certain company
claims the key to quality is one specific physicochemical property and
we find their information to be false, that company may also find their
business to be negatively impacted. With more awareness as to what makes
a quality wine, perceived high quality wine price could drop and
perceived low quality price could increase, if our results dispute their
perception.

Additionally, it is important to consider that our data is limited,
biased, and skewed. We cannot be sure that the data was collected in a
way that is accurate, precise, and equitable. Since all our data is from
one wine company, we are sure that all variations of wine in the world
and that all physicochemical properties affecting taste are not
represented in our dataset. With this in mind, if our results grow in
popularity, it can lead to positive or negative business for the company
who provided this data, depending on results. Lastly, we know that wine
quality is subjective and the ``rating'' portion of our data is biased.
With biased ratings, it is also likely that the results are skewed, with
more ratings at extreme ends than close to a median. With all this in
mind, we also consider that our results cannot be fully used to defend a
hypothesis, and that outside evidence is necessary for full
confirmation.

    \section{Conclusion \& Discussion}\label{conclusion-discussion}

    In conclusion, we found that though we are able to somewhat predict wine
quality, it is still not fine tuned enough to use ethically. Our
prediction is generally on par, sometimes slightly off, than the real
quality, within 2 at most. In relation to our hypothesis, all our
factors of physicochemical features combine to create different optimal
wines, meaning that all of our factors are involved in creating the
proper flavor profile. Our original hypothesis predicts that lower
volatile acidity, medium alcohol percentage, and higher citric acid
would create the best quality, and we did not necessarily find that to
be true. Interestingly, though we are already working with 12 variables,
there are still contributors to wine quality that are missing. With each
of the 12, we were unable to find one that stood out or a few that stood
out together in making a quality wine. They all needed to combine with
one another in a grand way to make a high quality wine.

A possible confound is in variables that cannot directly be measured
from the wine itself. Many experts cite the climate and soil at the
vineyard in which the wine originated, something that was not taken into
account with our data. This may account for part of the inaccuracy we
faced. Limitations in our data may also account for inaccuracy. As
stated before, we cannot be sure of the precision with which the data
was collected. Additionally, all of the wine in our dataset came from
the same vineyard, limiting our scope. Our dataset also lacked wine at
the extreme ends of high and low quality, limiting our training, thus
leading to inaccuracy. A few confounds are inevitable, but it may have
done us well to try to alleviate some of their effects.

If we were to continue this project, it may help to look deeper in to
advanced data science techniques for tackling problems like this. In our
pursuit to predict what makes a high quality wine, we found and learned
about KD Tree, which made sense of our data when we could not. There may
be more techniques out there that would've better our predictions and
help us properly pinpoint the best wine. Additionally, the addition of
more data may have helped us better find the correlations we were
looking for. Despite our attempts, the one true perfect wine is still
somewhat of a mystery, but we are sure data scientists will tackle it
better than we could.


    % Add a bibliography block to the postdoc
    
    
    
    \end{document}
